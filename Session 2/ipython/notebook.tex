
% Default to the notebook output style

    


% Inherit from the specified cell style.




    
\documentclass[11pt]{article}

    
    
    \usepackage[T1]{fontenc}
    % Nicer default font (+ math font) than Computer Modern for most use cases
    \usepackage{mathpazo}

    % Basic figure setup, for now with no caption control since it's done
    % automatically by Pandoc (which extracts ![](path) syntax from Markdown).
    \usepackage{graphicx}
    % We will generate all images so they have a width \maxwidth. This means
    % that they will get their normal width if they fit onto the page, but
    % are scaled down if they would overflow the margins.
    \makeatletter
    \def\maxwidth{\ifdim\Gin@nat@width>\linewidth\linewidth
    \else\Gin@nat@width\fi}
    \makeatother
    \let\Oldincludegraphics\includegraphics
    % Set max figure width to be 80% of text width, for now hardcoded.
    \renewcommand{\includegraphics}[1]{\Oldincludegraphics[width=.8\maxwidth]{#1}}
    % Ensure that by default, figures have no caption (until we provide a
    % proper Figure object with a Caption API and a way to capture that
    % in the conversion process - todo).
    \usepackage{caption}
    \DeclareCaptionLabelFormat{nolabel}{}
    \captionsetup{labelformat=nolabel}

    \usepackage{adjustbox} % Used to constrain images to a maximum size 
    \usepackage{xcolor} % Allow colors to be defined
    \usepackage{enumerate} % Needed for markdown enumerations to work
    \usepackage{geometry} % Used to adjust the document margins
    \usepackage{amsmath} % Equations
    \usepackage{amssymb} % Equations
    \usepackage{textcomp} % defines textquotesingle
    % Hack from http://tex.stackexchange.com/a/47451/13684:
    \AtBeginDocument{%
        \def\PYZsq{\textquotesingle}% Upright quotes in Pygmentized code
    }
    \usepackage{upquote} % Upright quotes for verbatim code
    \usepackage{eurosym} % defines \euro
    \usepackage[mathletters]{ucs} % Extended unicode (utf-8) support
    \usepackage[utf8x]{inputenc} % Allow utf-8 characters in the tex document
    \usepackage{fancyvrb} % verbatim replacement that allows latex
    \usepackage{grffile} % extends the file name processing of package graphics 
                         % to support a larger range 
    % The hyperref package gives us a pdf with properly built
    % internal navigation ('pdf bookmarks' for the table of contents,
    % internal cross-reference links, web links for URLs, etc.)
    \usepackage{hyperref}
    \usepackage{longtable} % longtable support required by pandoc >1.10
    \usepackage{booktabs}  % table support for pandoc > 1.12.2
    \usepackage[inline]{enumitem} % IRkernel/repr support (it uses the enumerate* environment)
    \usepackage[normalem]{ulem} % ulem is needed to support strikethroughs (\sout)
                                % normalem makes italics be italics, not underlines
    

    
    
    % Colors for the hyperref package
    \definecolor{urlcolor}{rgb}{0,.145,.698}
    \definecolor{linkcolor}{rgb}{.71,0.21,0.01}
    \definecolor{citecolor}{rgb}{.12,.54,.11}

    % ANSI colors
    \definecolor{ansi-black}{HTML}{3E424D}
    \definecolor{ansi-black-intense}{HTML}{282C36}
    \definecolor{ansi-red}{HTML}{E75C58}
    \definecolor{ansi-red-intense}{HTML}{B22B31}
    \definecolor{ansi-green}{HTML}{00A250}
    \definecolor{ansi-green-intense}{HTML}{007427}
    \definecolor{ansi-yellow}{HTML}{DDB62B}
    \definecolor{ansi-yellow-intense}{HTML}{B27D12}
    \definecolor{ansi-blue}{HTML}{208FFB}
    \definecolor{ansi-blue-intense}{HTML}{0065CA}
    \definecolor{ansi-magenta}{HTML}{D160C4}
    \definecolor{ansi-magenta-intense}{HTML}{A03196}
    \definecolor{ansi-cyan}{HTML}{60C6C8}
    \definecolor{ansi-cyan-intense}{HTML}{258F8F}
    \definecolor{ansi-white}{HTML}{C5C1B4}
    \definecolor{ansi-white-intense}{HTML}{A1A6B2}

    % commands and environments needed by pandoc snippets
    % extracted from the output of `pandoc -s`
    \providecommand{\tightlist}{%
      \setlength{\itemsep}{0pt}\setlength{\parskip}{0pt}}
    \DefineVerbatimEnvironment{Highlighting}{Verbatim}{commandchars=\\\{\}}
    % Add ',fontsize=\small' for more characters per line
    \newenvironment{Shaded}{}{}
    \newcommand{\KeywordTok}[1]{\textcolor[rgb]{0.00,0.44,0.13}{\textbf{{#1}}}}
    \newcommand{\DataTypeTok}[1]{\textcolor[rgb]{0.56,0.13,0.00}{{#1}}}
    \newcommand{\DecValTok}[1]{\textcolor[rgb]{0.25,0.63,0.44}{{#1}}}
    \newcommand{\BaseNTok}[1]{\textcolor[rgb]{0.25,0.63,0.44}{{#1}}}
    \newcommand{\FloatTok}[1]{\textcolor[rgb]{0.25,0.63,0.44}{{#1}}}
    \newcommand{\CharTok}[1]{\textcolor[rgb]{0.25,0.44,0.63}{{#1}}}
    \newcommand{\StringTok}[1]{\textcolor[rgb]{0.25,0.44,0.63}{{#1}}}
    \newcommand{\CommentTok}[1]{\textcolor[rgb]{0.38,0.63,0.69}{\textit{{#1}}}}
    \newcommand{\OtherTok}[1]{\textcolor[rgb]{0.00,0.44,0.13}{{#1}}}
    \newcommand{\AlertTok}[1]{\textcolor[rgb]{1.00,0.00,0.00}{\textbf{{#1}}}}
    \newcommand{\FunctionTok}[1]{\textcolor[rgb]{0.02,0.16,0.49}{{#1}}}
    \newcommand{\RegionMarkerTok}[1]{{#1}}
    \newcommand{\ErrorTok}[1]{\textcolor[rgb]{1.00,0.00,0.00}{\textbf{{#1}}}}
    \newcommand{\NormalTok}[1]{{#1}}
    
    % Additional commands for more recent versions of Pandoc
    \newcommand{\ConstantTok}[1]{\textcolor[rgb]{0.53,0.00,0.00}{{#1}}}
    \newcommand{\SpecialCharTok}[1]{\textcolor[rgb]{0.25,0.44,0.63}{{#1}}}
    \newcommand{\VerbatimStringTok}[1]{\textcolor[rgb]{0.25,0.44,0.63}{{#1}}}
    \newcommand{\SpecialStringTok}[1]{\textcolor[rgb]{0.73,0.40,0.53}{{#1}}}
    \newcommand{\ImportTok}[1]{{#1}}
    \newcommand{\DocumentationTok}[1]{\textcolor[rgb]{0.73,0.13,0.13}{\textit{{#1}}}}
    \newcommand{\AnnotationTok}[1]{\textcolor[rgb]{0.38,0.63,0.69}{\textbf{\textit{{#1}}}}}
    \newcommand{\CommentVarTok}[1]{\textcolor[rgb]{0.38,0.63,0.69}{\textbf{\textit{{#1}}}}}
    \newcommand{\VariableTok}[1]{\textcolor[rgb]{0.10,0.09,0.49}{{#1}}}
    \newcommand{\ControlFlowTok}[1]{\textcolor[rgb]{0.00,0.44,0.13}{\textbf{{#1}}}}
    \newcommand{\OperatorTok}[1]{\textcolor[rgb]{0.40,0.40,0.40}{{#1}}}
    \newcommand{\BuiltInTok}[1]{{#1}}
    \newcommand{\ExtensionTok}[1]{{#1}}
    \newcommand{\PreprocessorTok}[1]{\textcolor[rgb]{0.74,0.48,0.00}{{#1}}}
    \newcommand{\AttributeTok}[1]{\textcolor[rgb]{0.49,0.56,0.16}{{#1}}}
    \newcommand{\InformationTok}[1]{\textcolor[rgb]{0.38,0.63,0.69}{\textbf{\textit{{#1}}}}}
    \newcommand{\WarningTok}[1]{\textcolor[rgb]{0.38,0.63,0.69}{\textbf{\textit{{#1}}}}}
    
    
    % Define a nice break command that doesn't care if a line doesn't already
    % exist.
    \def\br{\hspace*{\fill} \\* }
    % Math Jax compatability definitions
    \def\gt{>}
    \def\lt{<}
    % Document parameters
    \title{Lesson 4 - Web API }
    
    
    

    % Pygments definitions
    
\makeatletter
\def\PY@reset{\let\PY@it=\relax \let\PY@bf=\relax%
    \let\PY@ul=\relax \let\PY@tc=\relax%
    \let\PY@bc=\relax \let\PY@ff=\relax}
\def\PY@tok#1{\csname PY@tok@#1\endcsname}
\def\PY@toks#1+{\ifx\relax#1\empty\else%
    \PY@tok{#1}\expandafter\PY@toks\fi}
\def\PY@do#1{\PY@bc{\PY@tc{\PY@ul{%
    \PY@it{\PY@bf{\PY@ff{#1}}}}}}}
\def\PY#1#2{\PY@reset\PY@toks#1+\relax+\PY@do{#2}}

\expandafter\def\csname PY@tok@w\endcsname{\def\PY@tc##1{\textcolor[rgb]{0.73,0.73,0.73}{##1}}}
\expandafter\def\csname PY@tok@c\endcsname{\let\PY@it=\textit\def\PY@tc##1{\textcolor[rgb]{0.25,0.50,0.50}{##1}}}
\expandafter\def\csname PY@tok@cp\endcsname{\def\PY@tc##1{\textcolor[rgb]{0.74,0.48,0.00}{##1}}}
\expandafter\def\csname PY@tok@k\endcsname{\let\PY@bf=\textbf\def\PY@tc##1{\textcolor[rgb]{0.00,0.50,0.00}{##1}}}
\expandafter\def\csname PY@tok@kp\endcsname{\def\PY@tc##1{\textcolor[rgb]{0.00,0.50,0.00}{##1}}}
\expandafter\def\csname PY@tok@kt\endcsname{\def\PY@tc##1{\textcolor[rgb]{0.69,0.00,0.25}{##1}}}
\expandafter\def\csname PY@tok@o\endcsname{\def\PY@tc##1{\textcolor[rgb]{0.40,0.40,0.40}{##1}}}
\expandafter\def\csname PY@tok@ow\endcsname{\let\PY@bf=\textbf\def\PY@tc##1{\textcolor[rgb]{0.67,0.13,1.00}{##1}}}
\expandafter\def\csname PY@tok@nb\endcsname{\def\PY@tc##1{\textcolor[rgb]{0.00,0.50,0.00}{##1}}}
\expandafter\def\csname PY@tok@nf\endcsname{\def\PY@tc##1{\textcolor[rgb]{0.00,0.00,1.00}{##1}}}
\expandafter\def\csname PY@tok@nc\endcsname{\let\PY@bf=\textbf\def\PY@tc##1{\textcolor[rgb]{0.00,0.00,1.00}{##1}}}
\expandafter\def\csname PY@tok@nn\endcsname{\let\PY@bf=\textbf\def\PY@tc##1{\textcolor[rgb]{0.00,0.00,1.00}{##1}}}
\expandafter\def\csname PY@tok@ne\endcsname{\let\PY@bf=\textbf\def\PY@tc##1{\textcolor[rgb]{0.82,0.25,0.23}{##1}}}
\expandafter\def\csname PY@tok@nv\endcsname{\def\PY@tc##1{\textcolor[rgb]{0.10,0.09,0.49}{##1}}}
\expandafter\def\csname PY@tok@no\endcsname{\def\PY@tc##1{\textcolor[rgb]{0.53,0.00,0.00}{##1}}}
\expandafter\def\csname PY@tok@nl\endcsname{\def\PY@tc##1{\textcolor[rgb]{0.63,0.63,0.00}{##1}}}
\expandafter\def\csname PY@tok@ni\endcsname{\let\PY@bf=\textbf\def\PY@tc##1{\textcolor[rgb]{0.60,0.60,0.60}{##1}}}
\expandafter\def\csname PY@tok@na\endcsname{\def\PY@tc##1{\textcolor[rgb]{0.49,0.56,0.16}{##1}}}
\expandafter\def\csname PY@tok@nt\endcsname{\let\PY@bf=\textbf\def\PY@tc##1{\textcolor[rgb]{0.00,0.50,0.00}{##1}}}
\expandafter\def\csname PY@tok@nd\endcsname{\def\PY@tc##1{\textcolor[rgb]{0.67,0.13,1.00}{##1}}}
\expandafter\def\csname PY@tok@s\endcsname{\def\PY@tc##1{\textcolor[rgb]{0.73,0.13,0.13}{##1}}}
\expandafter\def\csname PY@tok@sd\endcsname{\let\PY@it=\textit\def\PY@tc##1{\textcolor[rgb]{0.73,0.13,0.13}{##1}}}
\expandafter\def\csname PY@tok@si\endcsname{\let\PY@bf=\textbf\def\PY@tc##1{\textcolor[rgb]{0.73,0.40,0.53}{##1}}}
\expandafter\def\csname PY@tok@se\endcsname{\let\PY@bf=\textbf\def\PY@tc##1{\textcolor[rgb]{0.73,0.40,0.13}{##1}}}
\expandafter\def\csname PY@tok@sr\endcsname{\def\PY@tc##1{\textcolor[rgb]{0.73,0.40,0.53}{##1}}}
\expandafter\def\csname PY@tok@ss\endcsname{\def\PY@tc##1{\textcolor[rgb]{0.10,0.09,0.49}{##1}}}
\expandafter\def\csname PY@tok@sx\endcsname{\def\PY@tc##1{\textcolor[rgb]{0.00,0.50,0.00}{##1}}}
\expandafter\def\csname PY@tok@m\endcsname{\def\PY@tc##1{\textcolor[rgb]{0.40,0.40,0.40}{##1}}}
\expandafter\def\csname PY@tok@gh\endcsname{\let\PY@bf=\textbf\def\PY@tc##1{\textcolor[rgb]{0.00,0.00,0.50}{##1}}}
\expandafter\def\csname PY@tok@gu\endcsname{\let\PY@bf=\textbf\def\PY@tc##1{\textcolor[rgb]{0.50,0.00,0.50}{##1}}}
\expandafter\def\csname PY@tok@gd\endcsname{\def\PY@tc##1{\textcolor[rgb]{0.63,0.00,0.00}{##1}}}
\expandafter\def\csname PY@tok@gi\endcsname{\def\PY@tc##1{\textcolor[rgb]{0.00,0.63,0.00}{##1}}}
\expandafter\def\csname PY@tok@gr\endcsname{\def\PY@tc##1{\textcolor[rgb]{1.00,0.00,0.00}{##1}}}
\expandafter\def\csname PY@tok@ge\endcsname{\let\PY@it=\textit}
\expandafter\def\csname PY@tok@gs\endcsname{\let\PY@bf=\textbf}
\expandafter\def\csname PY@tok@gp\endcsname{\let\PY@bf=\textbf\def\PY@tc##1{\textcolor[rgb]{0.00,0.00,0.50}{##1}}}
\expandafter\def\csname PY@tok@go\endcsname{\def\PY@tc##1{\textcolor[rgb]{0.53,0.53,0.53}{##1}}}
\expandafter\def\csname PY@tok@gt\endcsname{\def\PY@tc##1{\textcolor[rgb]{0.00,0.27,0.87}{##1}}}
\expandafter\def\csname PY@tok@err\endcsname{\def\PY@bc##1{\setlength{\fboxsep}{0pt}\fcolorbox[rgb]{1.00,0.00,0.00}{1,1,1}{\strut ##1}}}
\expandafter\def\csname PY@tok@kc\endcsname{\let\PY@bf=\textbf\def\PY@tc##1{\textcolor[rgb]{0.00,0.50,0.00}{##1}}}
\expandafter\def\csname PY@tok@kd\endcsname{\let\PY@bf=\textbf\def\PY@tc##1{\textcolor[rgb]{0.00,0.50,0.00}{##1}}}
\expandafter\def\csname PY@tok@kn\endcsname{\let\PY@bf=\textbf\def\PY@tc##1{\textcolor[rgb]{0.00,0.50,0.00}{##1}}}
\expandafter\def\csname PY@tok@kr\endcsname{\let\PY@bf=\textbf\def\PY@tc##1{\textcolor[rgb]{0.00,0.50,0.00}{##1}}}
\expandafter\def\csname PY@tok@bp\endcsname{\def\PY@tc##1{\textcolor[rgb]{0.00,0.50,0.00}{##1}}}
\expandafter\def\csname PY@tok@fm\endcsname{\def\PY@tc##1{\textcolor[rgb]{0.00,0.00,1.00}{##1}}}
\expandafter\def\csname PY@tok@vc\endcsname{\def\PY@tc##1{\textcolor[rgb]{0.10,0.09,0.49}{##1}}}
\expandafter\def\csname PY@tok@vg\endcsname{\def\PY@tc##1{\textcolor[rgb]{0.10,0.09,0.49}{##1}}}
\expandafter\def\csname PY@tok@vi\endcsname{\def\PY@tc##1{\textcolor[rgb]{0.10,0.09,0.49}{##1}}}
\expandafter\def\csname PY@tok@vm\endcsname{\def\PY@tc##1{\textcolor[rgb]{0.10,0.09,0.49}{##1}}}
\expandafter\def\csname PY@tok@sa\endcsname{\def\PY@tc##1{\textcolor[rgb]{0.73,0.13,0.13}{##1}}}
\expandafter\def\csname PY@tok@sb\endcsname{\def\PY@tc##1{\textcolor[rgb]{0.73,0.13,0.13}{##1}}}
\expandafter\def\csname PY@tok@sc\endcsname{\def\PY@tc##1{\textcolor[rgb]{0.73,0.13,0.13}{##1}}}
\expandafter\def\csname PY@tok@dl\endcsname{\def\PY@tc##1{\textcolor[rgb]{0.73,0.13,0.13}{##1}}}
\expandafter\def\csname PY@tok@s2\endcsname{\def\PY@tc##1{\textcolor[rgb]{0.73,0.13,0.13}{##1}}}
\expandafter\def\csname PY@tok@sh\endcsname{\def\PY@tc##1{\textcolor[rgb]{0.73,0.13,0.13}{##1}}}
\expandafter\def\csname PY@tok@s1\endcsname{\def\PY@tc##1{\textcolor[rgb]{0.73,0.13,0.13}{##1}}}
\expandafter\def\csname PY@tok@mb\endcsname{\def\PY@tc##1{\textcolor[rgb]{0.40,0.40,0.40}{##1}}}
\expandafter\def\csname PY@tok@mf\endcsname{\def\PY@tc##1{\textcolor[rgb]{0.40,0.40,0.40}{##1}}}
\expandafter\def\csname PY@tok@mh\endcsname{\def\PY@tc##1{\textcolor[rgb]{0.40,0.40,0.40}{##1}}}
\expandafter\def\csname PY@tok@mi\endcsname{\def\PY@tc##1{\textcolor[rgb]{0.40,0.40,0.40}{##1}}}
\expandafter\def\csname PY@tok@il\endcsname{\def\PY@tc##1{\textcolor[rgb]{0.40,0.40,0.40}{##1}}}
\expandafter\def\csname PY@tok@mo\endcsname{\def\PY@tc##1{\textcolor[rgb]{0.40,0.40,0.40}{##1}}}
\expandafter\def\csname PY@tok@ch\endcsname{\let\PY@it=\textit\def\PY@tc##1{\textcolor[rgb]{0.25,0.50,0.50}{##1}}}
\expandafter\def\csname PY@tok@cm\endcsname{\let\PY@it=\textit\def\PY@tc##1{\textcolor[rgb]{0.25,0.50,0.50}{##1}}}
\expandafter\def\csname PY@tok@cpf\endcsname{\let\PY@it=\textit\def\PY@tc##1{\textcolor[rgb]{0.25,0.50,0.50}{##1}}}
\expandafter\def\csname PY@tok@c1\endcsname{\let\PY@it=\textit\def\PY@tc##1{\textcolor[rgb]{0.25,0.50,0.50}{##1}}}
\expandafter\def\csname PY@tok@cs\endcsname{\let\PY@it=\textit\def\PY@tc##1{\textcolor[rgb]{0.25,0.50,0.50}{##1}}}

\def\PYZbs{\char`\\}
\def\PYZus{\char`\_}
\def\PYZob{\char`\{}
\def\PYZcb{\char`\}}
\def\PYZca{\char`\^}
\def\PYZam{\char`\&}
\def\PYZlt{\char`\<}
\def\PYZgt{\char`\>}
\def\PYZsh{\char`\#}
\def\PYZpc{\char`\%}
\def\PYZdl{\char`\$}
\def\PYZhy{\char`\-}
\def\PYZsq{\char`\'}
\def\PYZdq{\char`\"}
\def\PYZti{\char`\~}
% for compatibility with earlier versions
\def\PYZat{@}
\def\PYZlb{[}
\def\PYZrb{]}
\makeatother


    % Exact colors from NB
    \definecolor{incolor}{rgb}{0.0, 0.0, 0.5}
    \definecolor{outcolor}{rgb}{0.545, 0.0, 0.0}



    
    % Prevent overflowing lines due to hard-to-break entities
    \sloppy 
    % Setup hyperref package
    \hypersetup{
      breaklinks=true,  % so long urls are correctly broken across lines
      colorlinks=true,
      urlcolor=urlcolor,
      linkcolor=linkcolor,
      citecolor=citecolor,
      }
    % Slightly bigger margins than the latex defaults
    
    \geometry{verbose,tmargin=1in,bmargin=1in,lmargin=1in,rmargin=1in}
    
    

    \begin{document}
    
    
    \maketitle
    
    

    
    \section{Lesson 4 - Web API}\label{lesson-4---web-api}

    \subsection{Requesting information from the
web}\label{requesting-information-from-the-web}

    \subsubsection{Python 'requests' module.}\label{python-requests-module.}

\begin{itemize}
\tightlist
\item
  This module provides functions to send a HTTP request and get the
  response from the server
\item
  Requests is a third party module. If not installed, we will need to do
  "pip install requests" in the mac terminal or in the command pronpt of
  windows.
\item
  http://docs.python-requests.org/en/master/user/quickstart/\#make-a-request
\end{itemize}

    \subsubsection{Using 'requests' module}\label{using-requests-module}

Use the requests module to make a HTTP request to
http://www.github.com/ibm - Check the status of the request - Display
the response header information

\begin{verbatim}
<img src="HTTPresponse.gif"  width="200" title="HTTP response packet">
\end{verbatim}

    \emph{Get status code for the request}

    \begin{Verbatim}[commandchars=\\\{\}]
{\color{incolor}In [{\color{incolor}1}]:} \PY{k+kn}{import} \PY{n+nn}{requests} 
        \PY{n}{url} \PY{o}{=} \PY{l+s+s1}{\PYZsq{}}\PY{l+s+s1}{http://www.github.com/ibm}\PY{l+s+s1}{\PYZsq{}}
        \PY{n}{response} \PY{o}{=} \PY{n}{requests}\PY{o}{.}\PY{n}{get}\PY{p}{(}\PY{n}{url}\PY{p}{)}
        
        \PY{n+nb}{print}\PY{p}{(}\PY{n}{response}\PY{o}{.}\PY{n}{status\PYZus{}code}\PY{p}{)}
\end{Verbatim}


    \begin{Verbatim}[commandchars=\\\{\}]
200

    \end{Verbatim}

    \emph{Get header information}

    \begin{Verbatim}[commandchars=\\\{\}]
{\color{incolor}In [{\color{incolor} }]:} \PY{k+kn}{import} \PY{n+nn}{requests} 
        \PY{n}{url} \PY{o}{=} \PY{l+s+s1}{\PYZsq{}}\PY{l+s+s1}{http://www.github.com/ibm}\PY{l+s+s1}{\PYZsq{}}
        \PY{n}{response} \PY{o}{=} \PY{n}{requests}\PY{o}{.}\PY{n}{get}\PY{p}{(}\PY{n}{url}\PY{p}{)}
        
        \PY{n+nb}{print}\PY{p}{(}\PY{n}{response}\PY{o}{.}\PY{n}{status\PYZus{}code}\PY{p}{)}
        
        \PY{k}{if} \PY{n}{response}\PY{o}{.}\PY{n}{status\PYZus{}code} \PY{o}{==} \PY{l+m+mi}{200}\PY{p}{:}
            \PY{n+nb}{print}\PY{p}{(}\PY{l+s+s1}{\PYZsq{}}\PY{l+s+s1}{Response status \PYZhy{} OK }\PY{l+s+s1}{\PYZsq{}}\PY{p}{)}
            \PY{n+nb}{print}\PY{p}{(}\PY{n}{response}\PY{o}{.}\PY{n}{headers}\PY{p}{)}
        \PY{k}{else}\PY{p}{:} 
            \PY{n+nb}{print}\PY{p}{(}\PY{l+s+s1}{\PYZsq{}}\PY{l+s+s1}{Error making the HTTP request }\PY{l+s+s1}{\PYZsq{}}\PY{p}{,}\PY{n}{response}\PY{o}{.}\PY{n}{status\PYZus{}code}  \PY{p}{)}
\end{Verbatim}


    \emph{Get the body Information}

    \begin{Verbatim}[commandchars=\\\{\}]
{\color{incolor}In [{\color{incolor} }]:} \PY{k+kn}{import} \PY{n+nn}{requests} 
        \PY{n}{url} \PY{o}{=} \PY{l+s+s1}{\PYZsq{}}\PY{l+s+s1}{http://www.github.com/ibm}\PY{l+s+s1}{\PYZsq{}}
        \PY{n}{response} \PY{o}{=} \PY{n}{requests}\PY{o}{.}\PY{n}{get}\PY{p}{(}\PY{n}{url}\PY{p}{)}
        
        \PY{n+nb}{print}\PY{p}{(}\PY{n}{response}\PY{o}{.}\PY{n}{status\PYZus{}code}\PY{p}{)}
        
        \PY{k}{if} \PY{n}{response}\PY{o}{.}\PY{n}{status\PYZus{}code} \PY{o}{==} \PY{l+m+mi}{200}\PY{p}{:}
            \PY{n+nb}{print}\PY{p}{(}\PY{l+s+s1}{\PYZsq{}}\PY{l+s+s1}{Response status \PYZhy{} OK }\PY{l+s+s1}{\PYZsq{}}\PY{p}{)}
            \PY{n+nb}{print}\PY{p}{(}\PY{n}{response}\PY{o}{.}\PY{n}{text}\PY{p}{)}
        \PY{k}{else}\PY{p}{:} 
            \PY{n+nb}{print}\PY{p}{(}\PY{l+s+s1}{\PYZsq{}}\PY{l+s+s1}{Error making the HTTP request }\PY{l+s+s1}{\PYZsq{}}\PY{p}{,}\PY{n}{response}\PY{o}{.}\PY{n}{status\PYZus{}code}  \PY{p}{)}
\end{Verbatim}


    \subsection{Using a Web API to Collect
Data}\label{using-a-web-api-to-collect-data}

\begin{itemize}
\tightlist
\item
  An application programming interface is a set of functions that you
  call to get access to some service.
\item
  An API is basically a list of functions and datatsructures for
  interfacting with websites's data.
\end{itemize}

The way these work is similar to viewing a web page. When you point your
browser to a website, you do it with a URL (http://www.github.com/ibm
for instance). Github sends you back data containing HTML, CSS, and
Javascript. Your browser uses this data to construct the page that you
see. The API works similarly, you request data with a URL
(http://api.github.com/org/ibm), but instead of getting HTML and such,
you get data formatted as JSON.

    \subsubsection{Access data using web
APIs}\label{access-data-using-web-apis}

\emph{Write a program to access all the public OSS projects hosted by
IBM on github.com using the web apis}

    \paragraph{Step 1: Access the Web API service and check rate
limits}\label{step-1-access-the-web-api-service-and-check-rate-limits}

    \begin{Verbatim}[commandchars=\\\{\}]
{\color{incolor}In [{\color{incolor}9}]:} \PY{k+kn}{import} \PY{n+nn}{requests}
            
        \PY{n}{url} \PY{o}{=} \PY{l+s+s2}{\PYZdq{}}\PY{l+s+s2}{https://api.github.com/orgs/ibm}\PY{l+s+s2}{\PYZdq{}}
        \PY{n}{response} \PY{o}{=} \PY{n}{requests}\PY{o}{.}\PY{n}{get}\PY{p}{(}\PY{n}{url}\PY{p}{)}
        
        \PY{k}{if} \PY{n}{response}\PY{o}{.}\PY{n}{status\PYZus{}code} \PY{o}{==} \PY{l+m+mi}{200}\PY{p}{:}
            \PY{n+nb}{print}\PY{p}{(}\PY{l+s+s1}{\PYZsq{}}\PY{l+s+s1}{Response status \PYZhy{} OK }\PY{l+s+s1}{\PYZsq{}}\PY{p}{)}
            \PY{n+nb}{print}\PY{p}{(}\PY{n}{response}\PY{o}{.}\PY{n}{headers}\PY{p}{[}\PY{l+s+s1}{\PYZsq{}}\PY{l+s+s1}{X\PYZhy{}RateLimit\PYZhy{}Remaining}\PY{l+s+s1}{\PYZsq{}}\PY{p}{]}\PY{p}{)}
        \PY{k}{else}\PY{p}{:}
            \PY{n+nb}{print}\PY{p}{(}\PY{l+s+s1}{\PYZsq{}}\PY{l+s+s1}{Error making the HTTP request }\PY{l+s+s1}{\PYZsq{}}\PY{p}{,}\PY{n}{response}\PY{o}{.}\PY{n}{status\PYZus{}code}  \PY{p}{)}
\end{Verbatim}


    \begin{Verbatim}[commandchars=\\\{\}]
Response status - OK 
53

    \end{Verbatim}

    \paragraph{Step 2: Authentication (if
required)}\label{step-2-authentication-if-required}

Authenticate requests to increase the API request limit. Access data
that requires authentication.

\subparagraph{Basic Authentication}\label{basic-authentication}

\begin{itemize}
\tightlist
\item
  Pass the userid and password as parameters in the response.get
  function
\item
  Little risky and prone to hacking. Create dummy user ID and password
\end{itemize}

\subparagraph{OAUTH}\label{oauth}

\begin{itemize}
\tightlist
\item
  OAuth 2 is an authorization framework that enables a user to connect
  to their account using a third party application
\item
  While this is more secure thant the basic authentication (i.e. passing
  the userid and password while you make a http request), it is a little
  more difficult to code.
\item
  It needs a personal token and a consumer key to be generated and
  passed to the webserver
\end{itemize}

Unfortunately different websites have different ways of generating and
using the token and consumer keys. Hence we will need to write the
authorization code for each website seperately. HOwever, every website
provides detailed information on how you can generate and send the token
and keys.

    \begin{Verbatim}[commandchars=\\\{\}]
{\color{incolor}In [{\color{incolor} }]:} \PY{k+kn}{import} \PY{n+nn}{requests}
            
        \PY{k}{def} \PY{n+nf}{GithubAPI}\PY{p}{(}\PY{n}{url}\PY{p}{)}\PY{p}{:}
            \PY{l+s+sd}{\PYZdq{}\PYZdq{}\PYZdq{} Make a HTTP request for the given URL and send the response body}
        \PY{l+s+sd}{    back to the calling function\PYZdq{}\PYZdq{}\PYZdq{}}
            \PY{c+c1}{\PYZsh{} Use basic authentication}
            \PY{n}{response} \PY{o}{=} \PY{n}{requests}\PY{o}{.}\PY{n}{get}\PY{p}{(}\PY{n}{url}\PY{p}{,} \PY{n}{auth}\PY{o}{=}\PY{p}{(}\PY{l+s+s2}{\PYZdq{}}\PY{l+s+s2}{ENTER USER ID}\PY{l+s+s2}{\PYZdq{}}\PY{p}{,}\PY{l+s+s2}{\PYZdq{}}\PY{l+s+s2}{ENTER PASSWORD}\PY{l+s+s2}{\PYZdq{}}\PY{p}{)}\PY{p}{)}
            \PY{k}{if} \PY{n}{response}\PY{o}{.}\PY{n}{status\PYZus{}code} \PY{o}{==} \PY{l+m+mi}{200}\PY{p}{:}
                \PY{n+nb}{print}\PY{p}{(}\PY{l+s+s1}{\PYZsq{}}\PY{l+s+s1}{Response status \PYZhy{} OK }\PY{l+s+s1}{\PYZsq{}}\PY{p}{)}
                \PY{n+nb}{print}\PY{p}{(}\PY{n}{response}\PY{o}{.}\PY{n}{headers}\PY{p}{[}\PY{l+s+s1}{\PYZsq{}}\PY{l+s+s1}{X\PYZhy{}RateLimit\PYZhy{}Remaining}\PY{l+s+s1}{\PYZsq{}}\PY{p}{]}\PY{p}{)}
                \PY{k}{return} \PY{n}{response}\PY{o}{.}\PY{n}{text}
            \PY{k}{else}\PY{p}{:}
                \PY{n+nb}{print}\PY{p}{(}\PY{l+s+s1}{\PYZsq{}}\PY{l+s+s1}{Error making the HTTP request }\PY{l+s+s1}{\PYZsq{}}\PY{p}{,}\PY{n}{response}\PY{o}{.}\PY{n}{status\PYZus{}code}  \PY{p}{)}
                \PY{k}{return} \PY{k+kc}{None}
        
        \PY{k}{def} \PY{n+nf}{main}\PY{p}{(}\PY{p}{)}\PY{p}{:}
            \PY{n}{url} \PY{o}{=} \PY{l+s+s2}{\PYZdq{}}\PY{l+s+s2}{https://api.github.com/orgs/ibm}\PY{l+s+s2}{\PYZdq{}}
            \PY{n}{txt\PYZus{}response} \PY{o}{=} \PY{n}{GithubAPI}\PY{p}{(}\PY{n}{url}\PY{p}{)}
            
            \PY{k}{if} \PY{n}{txt\PYZus{}response}\PY{p}{:}
                \PY{n+nb}{print}\PY{p}{(}\PY{n}{txt\PYZus{}response}\PY{p}{)}
        
        \PY{n}{main}\PY{p}{(}\PY{p}{)} 
\end{Verbatim}


    \paragraph{Step 3: Parse the response}\label{step-3-parse-the-response}

The \emph{json} module gives us functions to convert the JSON response
to a python readable data structure.

\emph{Write a program to get the number of OSS projects started by IBM}

    \begin{Verbatim}[commandchars=\\\{\}]
{\color{incolor}In [{\color{incolor}13}]:} \PY{k+kn}{import} \PY{n+nn}{requests}
         \PY{k+kn}{import} \PY{n+nn}{json}
         
         \PY{k}{def} \PY{n+nf}{GithubAPI}\PY{p}{(}\PY{n}{url}\PY{p}{)}\PY{p}{:}
             \PY{l+s+sd}{\PYZdq{}\PYZdq{}\PYZdq{} Make a HTTP request for the given URL and send the response body}
         \PY{l+s+sd}{    back to the calling function\PYZdq{}\PYZdq{}\PYZdq{}}
             \PY{n}{response} \PY{o}{=} \PY{n}{requests}\PY{o}{.}\PY{n}{get}\PY{p}{(}\PY{n}{url}\PY{p}{)}
             \PY{k}{if} \PY{n}{response}\PY{o}{.}\PY{n}{status\PYZus{}code} \PY{o}{==} \PY{l+m+mi}{200}\PY{p}{:}
                 \PY{n+nb}{print}\PY{p}{(}\PY{l+s+s1}{\PYZsq{}}\PY{l+s+s1}{Response status \PYZhy{} OK }\PY{l+s+s1}{\PYZsq{}}\PY{p}{)}
                 \PY{k}{return} \PY{n}{response}\PY{o}{.}\PY{n}{json}\PY{p}{(}\PY{p}{)}
             \PY{k}{else}\PY{p}{:}
                 \PY{n+nb}{print}\PY{p}{(}\PY{l+s+s1}{\PYZsq{}}\PY{l+s+s1}{Error making the HTTP request }\PY{l+s+s1}{\PYZsq{}}\PY{p}{,}\PY{n}{response}\PY{o}{.}\PY{n}{status\PYZus{}code}  \PY{p}{)}
                 \PY{k}{return} \PY{k+kc}{None}
         
         \PY{k}{def} \PY{n+nf}{main}\PY{p}{(}\PY{p}{)}\PY{p}{:}
             \PY{n}{url} \PY{o}{=} \PY{l+s+s2}{\PYZdq{}}\PY{l+s+s2}{https://api.github.com/orgs/ibm}\PY{l+s+s2}{\PYZdq{}}
             \PY{n}{txt\PYZus{}response} \PY{o}{=} \PY{n}{GithubAPI}\PY{p}{(}\PY{n}{url}\PY{p}{)}
             
             \PY{k}{if} \PY{n}{txt\PYZus{}response}\PY{p}{:}
                 \PY{n+nb}{print}\PY{p}{(}\PY{l+s+s1}{\PYZsq{}}\PY{l+s+s1}{The number of public repos are : }\PY{l+s+s1}{\PYZsq{}}\PY{p}{,}\PY{n}{txt\PYZus{}response}\PY{p}{[}\PY{l+s+s1}{\PYZsq{}}\PY{l+s+s1}{public\PYZus{}repos}\PY{l+s+s1}{\PYZsq{}}\PY{p}{]}\PY{p}{)}
         
         \PY{n}{main}\PY{p}{(}\PY{p}{)} 
\end{Verbatim}


    \begin{Verbatim}[commandchars=\\\{\}]
Response status - OK 
The number of public repos are :  851

    \end{Verbatim}

    \paragraph{Step 3: Follow the url information from the Web API to find
what you
need}\label{step-3-follow-the-url-information-from-the-web-api-to-find-what-you-need}

\emph{Let us collect the information regarding the different projects
started by IBM }

    \begin{Verbatim}[commandchars=\\\{\}]
{\color{incolor}In [{\color{incolor} }]:} \PY{k+kn}{import} \PY{n+nn}{requests}
        \PY{k+kn}{import} \PY{n+nn}{json}
                
        \PY{k}{def} \PY{n+nf}{GithubAPI}\PY{p}{(}\PY{n}{url}\PY{p}{)}\PY{p}{:}
            \PY{l+s+sd}{\PYZdq{}\PYZdq{}\PYZdq{} Make a HTTP request for the given URL and send the response body}
        \PY{l+s+sd}{    back to the calling function\PYZdq{}\PYZdq{}\PYZdq{}}
            \PY{n}{response} \PY{o}{=} \PY{n}{requests}\PY{o}{.}\PY{n}{get}\PY{p}{(}\PY{n}{url}\PY{p}{,} \PY{n}{auth}\PY{p}{(}\PY{l+s+s2}{\PYZdq{}}\PY{l+s+s2}{ENTER USER ID}\PY{l+s+s2}{\PYZdq{}}\PY{p}{,}\PY{l+s+s2}{\PYZdq{}}\PY{l+s+s2}{ENTER PASSWORD}\PY{l+s+s2}{\PYZdq{}}\PY{p}{)}\PY{p}{)}
            \PY{k}{if} \PY{n}{response}\PY{o}{.}\PY{n}{status\PYZus{}code} \PY{o}{==} \PY{l+m+mi}{200}\PY{p}{:}
                \PY{n+nb}{print}\PY{p}{(}\PY{l+s+s1}{\PYZsq{}}\PY{l+s+s1}{Response status \PYZhy{} OK }\PY{l+s+s1}{\PYZsq{}}\PY{p}{)}
                \PY{k}{return} \PY{n}{response}\PY{o}{.}\PY{n}{json}\PY{p}{(}\PY{p}{)}
            \PY{k}{else}\PY{p}{:}
                \PY{n+nb}{print}\PY{p}{(}\PY{l+s+s1}{\PYZsq{}}\PY{l+s+s1}{Error making the HTTP request }\PY{l+s+s1}{\PYZsq{}}\PY{p}{,}\PY{n}{response}\PY{o}{.}\PY{n}{status\PYZus{}code}  \PY{p}{)}
                \PY{k}{return} \PY{k+kc}{None}
        
        \PY{k}{def} \PY{n+nf}{main}\PY{p}{(}\PY{p}{)}\PY{p}{:}
            \PY{n}{url} \PY{o}{=} \PY{l+s+s2}{\PYZdq{}}\PY{l+s+s2}{https://api.github.com/orgs/ibm}\PY{l+s+s2}{\PYZdq{}}
            \PY{n}{response\PYZus{}json} \PY{o}{=} \PY{n}{GithubAPI}\PY{p}{(}\PY{n}{url}\PY{p}{)}
            
            \PY{k}{if} \PY{n}{response\PYZus{}json}\PY{p}{:}
                \PY{n+nb}{print}\PY{p}{(}\PY{l+s+s1}{\PYZsq{}}\PY{l+s+s1}{The number of public repos are : }\PY{l+s+s1}{\PYZsq{}}\PY{p}{,}\PY{n}{response\PYZus{}json}\PY{p}{[}\PY{l+s+s1}{\PYZsq{}}\PY{l+s+s1}{public\PYZus{}repos}\PY{l+s+s1}{\PYZsq{}}\PY{p}{]}\PY{p}{)}
                \PY{n}{repo\PYZus{}url} \PY{o}{=} \PY{n}{response\PYZus{}json}\PY{p}{[}\PY{l+s+s1}{\PYZsq{}}\PY{l+s+s1}{repos\PYZus{}url}\PY{l+s+s1}{\PYZsq{}}\PY{p}{]}
                \PY{n}{repo\PYZus{}response} \PY{o}{=} \PY{n}{GithubAPI}\PY{p}{(}\PY{n}{repo\PYZus{}url}\PY{p}{)}
                \PY{k}{for} \PY{n}{repo} \PY{o+ow}{in} \PY{n}{repo\PYZus{}response}\PY{p}{:}
                    \PY{n+nb}{print}\PY{p}{(}\PY{p}{[}\PY{n}{repo}\PY{p}{[}\PY{l+s+s1}{\PYZsq{}}\PY{l+s+s1}{id}\PY{l+s+s1}{\PYZsq{}}\PY{p}{]}\PY{p}{,}\PY{n}{repo}\PY{p}{[}\PY{l+s+s1}{\PYZsq{}}\PY{l+s+s1}{name}\PY{l+s+s1}{\PYZsq{}}\PY{p}{]}\PY{p}{]}\PY{p}{)}
                
        \PY{n}{main}\PY{p}{(}\PY{p}{)} 
\end{Verbatim}


    \paragraph{Step 4: Paginate to get data from other
pages}\label{step-4-paginate-to-get-data-from-other-pages}

\emph{Traverse the pages if the data is spread across multiple pages}

    \begin{Verbatim}[commandchars=\\\{\}]
{\color{incolor}In [{\color{incolor} }]:} \PY{k+kn}{import} \PY{n+nn}{requests}
        \PY{k+kn}{import} \PY{n+nn}{json}
                
        \PY{k}{def} \PY{n+nf}{GithubAPI}\PY{p}{(}\PY{n}{url}\PY{p}{)}\PY{p}{:}
            \PY{l+s+sd}{\PYZdq{}\PYZdq{}\PYZdq{} Make a HTTP request for the given URL and send the response body}
        \PY{l+s+sd}{    back to the calling function\PYZdq{}\PYZdq{}\PYZdq{}}
            \PY{n}{response} \PY{o}{=} \PY{n}{requests}\PY{o}{.}\PY{n}{get}\PY{p}{(}\PY{n}{url}\PY{p}{,} \PY{n}{auth} \PY{o}{=} \PY{p}{(}\PY{l+s+s2}{\PYZdq{}}\PY{l+s+s2}{ENTER USER ID}\PY{l+s+s2}{\PYZdq{}}\PY{p}{,}\PY{l+s+s2}{\PYZdq{}}\PY{l+s+s2}{ENTER PASSWORD}\PY{l+s+s2}{\PYZdq{}}\PY{p}{)}\PY{p}{)}
            \PY{k}{if} \PY{n}{response}\PY{o}{.}\PY{n}{status\PYZus{}code} \PY{o}{==} \PY{l+m+mi}{200}\PY{p}{:}
                \PY{n+nb}{print}\PY{p}{(}\PY{l+s+s1}{\PYZsq{}}\PY{l+s+s1}{Response status \PYZhy{} OK }\PY{l+s+s1}{\PYZsq{}}\PY{p}{)}
                \PY{k}{return} \PY{n}{response}\PY{o}{.}\PY{n}{json}\PY{p}{(}\PY{p}{)}
            \PY{k}{else}\PY{p}{:}
                \PY{n+nb}{print}\PY{p}{(}\PY{l+s+s1}{\PYZsq{}}\PY{l+s+s1}{Error making the HTTP request }\PY{l+s+s1}{\PYZsq{}}\PY{p}{,}\PY{n}{response}\PY{o}{.}\PY{n}{status\PYZus{}code}  \PY{p}{)}
                \PY{k}{return} \PY{k+kc}{None}
        
        \PY{k}{def} \PY{n+nf}{main}\PY{p}{(}\PY{p}{)}\PY{p}{:}
            \PY{n}{url} \PY{o}{=} \PY{l+s+s2}{\PYZdq{}}\PY{l+s+s2}{https://api.github.com/orgs/ibm}\PY{l+s+s2}{\PYZdq{}}
            \PY{n}{response\PYZus{}json} \PY{o}{=} \PY{n}{GithubAPI}\PY{p}{(}\PY{n}{url}\PY{p}{)}
            
            \PY{k}{if} \PY{n}{response\PYZus{}json}\PY{p}{:}
                \PY{n+nb}{print}\PY{p}{(}\PY{l+s+s1}{\PYZsq{}}\PY{l+s+s1}{The number of public repos are : }\PY{l+s+s1}{\PYZsq{}}\PY{p}{,}\PY{n}{response\PYZus{}json}\PY{p}{[}\PY{l+s+s1}{\PYZsq{}}\PY{l+s+s1}{public\PYZus{}repos}\PY{l+s+s1}{\PYZsq{}}\PY{p}{]}\PY{p}{)}
                \PY{n}{repo\PYZus{}url} \PY{o}{=} \PY{n}{response\PYZus{}json}\PY{p}{[}\PY{l+s+s1}{\PYZsq{}}\PY{l+s+s1}{repos\PYZus{}url}\PY{l+s+s1}{\PYZsq{}}\PY{p}{]}
                \PY{n}{total\PYZus{}no} \PY{o}{=}  \PY{n}{response\PYZus{}json}\PY{p}{[}\PY{l+s+s1}{\PYZsq{}}\PY{l+s+s1}{public\PYZus{}repos}\PY{l+s+s1}{\PYZsq{}}\PY{p}{]}
                \PY{n}{per\PYZus{}page} \PY{o}{=} \PY{l+m+mi}{100}
                \PY{n}{page\PYZus{}count} \PY{o}{=} \PY{l+m+mi}{1}
                \PY{k}{while} \PY{n}{page\PYZus{}count} \PY{o}{\PYZlt{}} \PY{n}{total\PYZus{}no}\PY{o}{/}\PY{l+m+mi}{100}\PY{p}{:}
                    \PY{c+c1}{\PYZsh{}Display 100 repos per page and traverse the pages untill we get the last page}
                    \PY{n}{page\PYZus{}url} \PY{o}{=} \PY{n}{repo\PYZus{}url}\PY{o}{+}\PY{l+s+s2}{\PYZdq{}}\PY{l+s+s2}{?per\PYZus{}page=100\PYZam{}page\PYZus{}no=}\PY{l+s+s2}{\PYZdq{}}\PY{o}{+}\PY{n+nb}{str}\PY{p}{(}\PY{n}{page\PYZus{}count}\PY{p}{)}
                    \PY{n+nb}{print}\PY{p}{(}\PY{n}{page\PYZus{}url}\PY{p}{)}
                    \PY{n}{repo\PYZus{}response} \PY{o}{=} \PY{n}{GithubAPI}\PY{p}{(}\PY{n}{page\PYZus{}url}\PY{p}{)}
                    \PY{c+c1}{\PYZsh{} Increment page number}
                    \PY{n}{page\PYZus{}count} \PY{o}{=} \PY{n}{page\PYZus{}count}\PY{o}{+}\PY{l+m+mi}{1}
                    \PY{k}{for} \PY{n}{repo} \PY{o+ow}{in} \PY{n}{repo\PYZus{}response}\PY{p}{:}
                        \PY{n+nb}{print}\PY{p}{(}\PY{p}{[}\PY{n}{repo}\PY{p}{[}\PY{l+s+s1}{\PYZsq{}}\PY{l+s+s1}{id}\PY{l+s+s1}{\PYZsq{}}\PY{p}{]}\PY{p}{,}\PY{n}{repo}\PY{p}{[}\PY{l+s+s1}{\PYZsq{}}\PY{l+s+s1}{name}\PY{l+s+s1}{\PYZsq{}}\PY{p}{]}\PY{p}{]}\PY{p}{)}
                        
        \PY{n}{main}\PY{p}{(}\PY{p}{)} 
\end{Verbatim}


    \subsection{Write a CSV}\label{write-a-csv}

Lets try to write the repos into a CSV file.

    \emph{Write a code to append data row wise to a csv file}

    \begin{Verbatim}[commandchars=\\\{\}]
{\color{incolor}In [{\color{incolor} }]:} \PY{k+kn}{import} \PY{n+nn}{csv}
        \PY{n}{WRITE\PYZus{}CSV} \PY{o}{=} \PY{l+s+s2}{\PYZdq{}}\PY{l+s+s2}{C:/Users/kmpoo/Dropbox/HEC/Teaching/Python for PhD Mar 2018/python4phd/Session 2/ipython/Repo\PYZus{}csv.csv}\PY{l+s+s2}{\PYZdq{}}
        \PY{k}{with} \PY{n+nb}{open}\PY{p}{(}\PY{n}{WRITE\PYZus{}CSV}\PY{p}{,} \PY{l+s+s1}{\PYZsq{}}\PY{l+s+s1}{at}\PY{l+s+s1}{\PYZsq{}}\PY{p}{,}\PY{n}{encoding} \PY{o}{=} \PY{l+s+s1}{\PYZsq{}}\PY{l+s+s1}{utf\PYZhy{}8}\PY{l+s+s1}{\PYZsq{}}\PY{p}{,} \PY{n}{newline}\PY{o}{=}\PY{l+s+s1}{\PYZsq{}}\PY{l+s+s1}{\PYZsq{}}\PY{p}{)} \PY{k}{as} \PY{n}{csv\PYZus{}obj}\PY{p}{:}
            \PY{n}{write} \PY{o}{=} \PY{n}{csv}\PY{o}{.}\PY{n}{writer}\PY{p}{(}\PY{n}{csv\PYZus{}obj}\PY{p}{)} \PY{c+c1}{\PYZsh{} Note it is csv.writer not reader}
            
            \PY{n}{write}\PY{o}{.}\PY{n}{writerow}\PY{p}{(}\PY{p}{[}\PY{l+s+s1}{\PYZsq{}}\PY{l+s+s1}{REPO ID}\PY{l+s+s1}{\PYZsq{}}\PY{p}{,}\PY{l+s+s1}{\PYZsq{}}\PY{l+s+s1}{REPO NAME}\PY{l+s+s1}{\PYZsq{}}\PY{p}{]}\PY{p}{)}
\end{Verbatim}


    \emph{What do you think will happen if we use 'wt' as mode instead of
'at' ?}

    \emph{Write a program so that you save the IBM repositories into the CSV
file. So that each row is a new repository and column 1 is the ID and
column 2 is the name}

    \begin{Verbatim}[commandchars=\\\{\}]
{\color{incolor}In [{\color{incolor} }]:} \PY{c+c1}{\PYZsh{}Enter code here}
        
        \PY{k+kn}{import} \PY{n+nn}{requests}
        \PY{k+kn}{import} \PY{n+nn}{json}
        \PY{k+kn}{import} \PY{n+nn}{csv}
        
        \PY{n}{WRITE\PYZus{}CSV} \PY{o}{=} \PY{l+s+s2}{\PYZdq{}}\PY{l+s+s2}{C:/Users/kmpoo/Dropbox/HEC/Teaching/Python for PhD Mar 2018/python4phd/Session 2/ipython/Repo\PYZus{}csv.csv}\PY{l+s+s2}{\PYZdq{}}
        
        \PY{k}{def} \PY{n+nf}{appendcsv}\PY{p}{(}\PY{n}{data\PYZus{}list}\PY{p}{)}\PY{p}{:}
            \PY{k}{with} \PY{n+nb}{open}\PY{p}{(}\PY{n}{WRITE\PYZus{}CSV}\PY{p}{,} \PY{l+s+s1}{\PYZsq{}}\PY{l+s+s1}{at}\PY{l+s+s1}{\PYZsq{}}\PY{p}{,}\PY{n}{encoding} \PY{o}{=} \PY{l+s+s1}{\PYZsq{}}\PY{l+s+s1}{utf\PYZhy{}8}\PY{l+s+s1}{\PYZsq{}}\PY{p}{,} \PY{n}{newline}\PY{o}{=}\PY{l+s+s1}{\PYZsq{}}\PY{l+s+s1}{\PYZsq{}}\PY{p}{)} \PY{k}{as} \PY{n}{csv\PYZus{}obj}\PY{p}{:}
                \PY{n}{write} \PY{o}{=} \PY{n}{csv}\PY{o}{.}\PY{n}{writer}\PY{p}{(}\PY{n}{csv\PYZus{}obj}\PY{p}{)} \PY{c+c1}{\PYZsh{} Note it is csv.writer not reader }
                \PY{n}{write}\PY{o}{.}\PY{n}{writerow}\PY{p}{(}\PY{n}{data\PYZus{}list}\PY{p}{)}
                
        \PY{k}{def} \PY{n+nf}{GithubAPI}\PY{p}{(}\PY{n}{url}\PY{p}{)}\PY{p}{:}
            \PY{l+s+sd}{\PYZdq{}\PYZdq{}\PYZdq{} Make a HTTP request for the given URL and send the response body}
        \PY{l+s+sd}{    back to the calling function\PYZdq{}\PYZdq{}\PYZdq{}}
            \PY{n}{response} \PY{o}{=} \PY{n}{requests}\PY{o}{.}\PY{n}{get}\PY{p}{(}\PY{n}{url}\PY{p}{,} \PY{n}{auth} \PY{o}{=} \PY{p}{(}\PY{l+s+s2}{\PYZdq{}}\PY{l+s+s2}{ENTER USER ID}\PY{l+s+s2}{\PYZdq{}}\PY{p}{,}\PY{l+s+s2}{\PYZdq{}}\PY{l+s+s2}{ENTER PASSWORD}\PY{l+s+s2}{\PYZdq{}}\PY{p}{)}\PY{p}{)}
            \PY{k}{if} \PY{n}{response}\PY{o}{.}\PY{n}{status\PYZus{}code} \PY{o}{==} \PY{l+m+mi}{200}\PY{p}{:}
                \PY{n+nb}{print}\PY{p}{(}\PY{l+s+s1}{\PYZsq{}}\PY{l+s+s1}{Response status \PYZhy{} OK }\PY{l+s+s1}{\PYZsq{}}\PY{p}{)}
                \PY{k}{return} \PY{n}{response}\PY{o}{.}\PY{n}{json}\PY{p}{(}\PY{p}{)}
            \PY{k}{else}\PY{p}{:}
                \PY{n+nb}{print}\PY{p}{(}\PY{l+s+s1}{\PYZsq{}}\PY{l+s+s1}{Error making the HTTP request }\PY{l+s+s1}{\PYZsq{}}\PY{p}{,}\PY{n}{response}\PY{o}{.}\PY{n}{status\PYZus{}code}  \PY{p}{)}
                \PY{k}{return} \PY{k+kc}{None}
        
        \PY{k}{def} \PY{n+nf}{main}\PY{p}{(}\PY{p}{)}\PY{p}{:}
            \PY{n}{url} \PY{o}{=} \PY{l+s+s2}{\PYZdq{}}\PY{l+s+s2}{https://api.github.com/orgs/ibm}\PY{l+s+s2}{\PYZdq{}}
            \PY{n}{response\PYZus{}json} \PY{o}{=} \PY{n}{GithubAPI}\PY{p}{(}\PY{n}{url}\PY{p}{)}
            
            \PY{k}{if} \PY{n}{response\PYZus{}json}\PY{p}{:}
                \PY{n+nb}{print}\PY{p}{(}\PY{l+s+s1}{\PYZsq{}}\PY{l+s+s1}{The number of public repos are : }\PY{l+s+s1}{\PYZsq{}}\PY{p}{,}\PY{n}{response\PYZus{}json}\PY{p}{[}\PY{l+s+s1}{\PYZsq{}}\PY{l+s+s1}{public\PYZus{}repos}\PY{l+s+s1}{\PYZsq{}}\PY{p}{]}\PY{p}{)}
                \PY{n}{repo\PYZus{}url} \PY{o}{=} \PY{n}{response\PYZus{}json}\PY{p}{[}\PY{l+s+s1}{\PYZsq{}}\PY{l+s+s1}{repos\PYZus{}url}\PY{l+s+s1}{\PYZsq{}}\PY{p}{]}
                \PY{n}{total\PYZus{}no} \PY{o}{=}  \PY{n}{response\PYZus{}json}\PY{p}{[}\PY{l+s+s1}{\PYZsq{}}\PY{l+s+s1}{public\PYZus{}repos}\PY{l+s+s1}{\PYZsq{}}\PY{p}{]}
                \PY{n}{per\PYZus{}page} \PY{o}{=} \PY{l+m+mi}{100}
                \PY{n}{page\PYZus{}count} \PY{o}{=} \PY{l+m+mi}{1}
                \PY{k}{while} \PY{n}{page\PYZus{}count} \PY{o}{\PYZlt{}} \PY{n}{total\PYZus{}no}\PY{o}{/}\PY{l+m+mi}{100}\PY{p}{:}
                    \PY{c+c1}{\PYZsh{}Display 100 repos per page and traverse the pages untill we get the last page}
                    \PY{n}{page\PYZus{}url} \PY{o}{=} \PY{n}{repo\PYZus{}url}\PY{o}{+}\PY{l+s+s2}{\PYZdq{}}\PY{l+s+s2}{?per\PYZus{}page=100\PYZam{}page\PYZus{}no=}\PY{l+s+s2}{\PYZdq{}}\PY{o}{+}\PY{n+nb}{str}\PY{p}{(}\PY{n}{page\PYZus{}count}\PY{p}{)}
                    \PY{n+nb}{print}\PY{p}{(}\PY{n}{page\PYZus{}url}\PY{p}{)}
                    \PY{n}{repo\PYZus{}response} \PY{o}{=} \PY{n}{GithubAPI}\PY{p}{(}\PY{n}{page\PYZus{}url}\PY{p}{)}
                    \PY{c+c1}{\PYZsh{} Increment page number}
                    \PY{n}{page\PYZus{}count} \PY{o}{=} \PY{n}{page\PYZus{}count}\PY{o}{+}\PY{l+m+mi}{1}
                    \PY{k}{for} \PY{n}{repo} \PY{o+ow}{in} \PY{n}{repo\PYZus{}response}\PY{p}{:}
                        \PY{n+nb}{print}\PY{p}{(}\PY{p}{[}\PY{n}{repo}\PY{p}{[}\PY{l+s+s1}{\PYZsq{}}\PY{l+s+s1}{id}\PY{l+s+s1}{\PYZsq{}}\PY{p}{]}\PY{p}{,}\PY{n}{repo}\PY{p}{[}\PY{l+s+s1}{\PYZsq{}}\PY{l+s+s1}{name}\PY{l+s+s1}{\PYZsq{}}\PY{p}{]}\PY{p}{]}\PY{p}{)}
                        \PY{n}{appendcsv}\PY{p}{(}\PY{p}{[}\PY{n}{repo}\PY{p}{[}\PY{l+s+s1}{\PYZsq{}}\PY{l+s+s1}{id}\PY{l+s+s1}{\PYZsq{}}\PY{p}{]}\PY{p}{,}\PY{n}{repo}\PY{p}{[}\PY{l+s+s1}{\PYZsq{}}\PY{l+s+s1}{name}\PY{l+s+s1}{\PYZsq{}}\PY{p}{]}\PY{p}{]}\PY{p}{)}
                        
                        
        \PY{n}{main}\PY{p}{(}\PY{p}{)} 
\end{Verbatim}



    % Add a bibliography block to the postdoc
    
    
    
    \end{document}
