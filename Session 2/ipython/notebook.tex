
% Default to the notebook output style

    


% Inherit from the specified cell style.




    
\documentclass[11pt]{article}

    
    
    \usepackage[T1]{fontenc}
    % Nicer default font (+ math font) than Computer Modern for most use cases
    \usepackage{mathpazo}

    % Basic figure setup, for now with no caption control since it's done
    % automatically by Pandoc (which extracts ![](path) syntax from Markdown).
    \usepackage{graphicx}
    % We will generate all images so they have a width \maxwidth. This means
    % that they will get their normal width if they fit onto the page, but
    % are scaled down if they would overflow the margins.
    \makeatletter
    \def\maxwidth{\ifdim\Gin@nat@width>\linewidth\linewidth
    \else\Gin@nat@width\fi}
    \makeatother
    \let\Oldincludegraphics\includegraphics
    % Set max figure width to be 80% of text width, for now hardcoded.
    \renewcommand{\includegraphics}[1]{\Oldincludegraphics[width=.8\maxwidth]{#1}}
    % Ensure that by default, figures have no caption (until we provide a
    % proper Figure object with a Caption API and a way to capture that
    % in the conversion process - todo).
    \usepackage{caption}
    \DeclareCaptionLabelFormat{nolabel}{}
    \captionsetup{labelformat=nolabel}

    \usepackage{adjustbox} % Used to constrain images to a maximum size 
    \usepackage{xcolor} % Allow colors to be defined
    \usepackage{enumerate} % Needed for markdown enumerations to work
    \usepackage{geometry} % Used to adjust the document margins
    \usepackage{amsmath} % Equations
    \usepackage{amssymb} % Equations
    \usepackage{textcomp} % defines textquotesingle
    % Hack from http://tex.stackexchange.com/a/47451/13684:
    \AtBeginDocument{%
        \def\PYZsq{\textquotesingle}% Upright quotes in Pygmentized code
    }
    \usepackage{upquote} % Upright quotes for verbatim code
    \usepackage{eurosym} % defines \euro
    \usepackage[mathletters]{ucs} % Extended unicode (utf-8) support
    \usepackage[utf8x]{inputenc} % Allow utf-8 characters in the tex document
    \usepackage{fancyvrb} % verbatim replacement that allows latex
    \usepackage{grffile} % extends the file name processing of package graphics 
                         % to support a larger range 
    % The hyperref package gives us a pdf with properly built
    % internal navigation ('pdf bookmarks' for the table of contents,
    % internal cross-reference links, web links for URLs, etc.)
    \usepackage{hyperref}
    \usepackage{longtable} % longtable support required by pandoc >1.10
    \usepackage{booktabs}  % table support for pandoc > 1.12.2
    \usepackage[inline]{enumitem} % IRkernel/repr support (it uses the enumerate* environment)
    \usepackage[normalem]{ulem} % ulem is needed to support strikethroughs (\sout)
                                % normalem makes italics be italics, not underlines
    

    
    
    % Colors for the hyperref package
    \definecolor{urlcolor}{rgb}{0,.145,.698}
    \definecolor{linkcolor}{rgb}{.71,0.21,0.01}
    \definecolor{citecolor}{rgb}{.12,.54,.11}

    % ANSI colors
    \definecolor{ansi-black}{HTML}{3E424D}
    \definecolor{ansi-black-intense}{HTML}{282C36}
    \definecolor{ansi-red}{HTML}{E75C58}
    \definecolor{ansi-red-intense}{HTML}{B22B31}
    \definecolor{ansi-green}{HTML}{00A250}
    \definecolor{ansi-green-intense}{HTML}{007427}
    \definecolor{ansi-yellow}{HTML}{DDB62B}
    \definecolor{ansi-yellow-intense}{HTML}{B27D12}
    \definecolor{ansi-blue}{HTML}{208FFB}
    \definecolor{ansi-blue-intense}{HTML}{0065CA}
    \definecolor{ansi-magenta}{HTML}{D160C4}
    \definecolor{ansi-magenta-intense}{HTML}{A03196}
    \definecolor{ansi-cyan}{HTML}{60C6C8}
    \definecolor{ansi-cyan-intense}{HTML}{258F8F}
    \definecolor{ansi-white}{HTML}{C5C1B4}
    \definecolor{ansi-white-intense}{HTML}{A1A6B2}

    % commands and environments needed by pandoc snippets
    % extracted from the output of `pandoc -s`
    \providecommand{\tightlist}{%
      \setlength{\itemsep}{0pt}\setlength{\parskip}{0pt}}
    \DefineVerbatimEnvironment{Highlighting}{Verbatim}{commandchars=\\\{\}}
    % Add ',fontsize=\small' for more characters per line
    \newenvironment{Shaded}{}{}
    \newcommand{\KeywordTok}[1]{\textcolor[rgb]{0.00,0.44,0.13}{\textbf{{#1}}}}
    \newcommand{\DataTypeTok}[1]{\textcolor[rgb]{0.56,0.13,0.00}{{#1}}}
    \newcommand{\DecValTok}[1]{\textcolor[rgb]{0.25,0.63,0.44}{{#1}}}
    \newcommand{\BaseNTok}[1]{\textcolor[rgb]{0.25,0.63,0.44}{{#1}}}
    \newcommand{\FloatTok}[1]{\textcolor[rgb]{0.25,0.63,0.44}{{#1}}}
    \newcommand{\CharTok}[1]{\textcolor[rgb]{0.25,0.44,0.63}{{#1}}}
    \newcommand{\StringTok}[1]{\textcolor[rgb]{0.25,0.44,0.63}{{#1}}}
    \newcommand{\CommentTok}[1]{\textcolor[rgb]{0.38,0.63,0.69}{\textit{{#1}}}}
    \newcommand{\OtherTok}[1]{\textcolor[rgb]{0.00,0.44,0.13}{{#1}}}
    \newcommand{\AlertTok}[1]{\textcolor[rgb]{1.00,0.00,0.00}{\textbf{{#1}}}}
    \newcommand{\FunctionTok}[1]{\textcolor[rgb]{0.02,0.16,0.49}{{#1}}}
    \newcommand{\RegionMarkerTok}[1]{{#1}}
    \newcommand{\ErrorTok}[1]{\textcolor[rgb]{1.00,0.00,0.00}{\textbf{{#1}}}}
    \newcommand{\NormalTok}[1]{{#1}}
    
    % Additional commands for more recent versions of Pandoc
    \newcommand{\ConstantTok}[1]{\textcolor[rgb]{0.53,0.00,0.00}{{#1}}}
    \newcommand{\SpecialCharTok}[1]{\textcolor[rgb]{0.25,0.44,0.63}{{#1}}}
    \newcommand{\VerbatimStringTok}[1]{\textcolor[rgb]{0.25,0.44,0.63}{{#1}}}
    \newcommand{\SpecialStringTok}[1]{\textcolor[rgb]{0.73,0.40,0.53}{{#1}}}
    \newcommand{\ImportTok}[1]{{#1}}
    \newcommand{\DocumentationTok}[1]{\textcolor[rgb]{0.73,0.13,0.13}{\textit{{#1}}}}
    \newcommand{\AnnotationTok}[1]{\textcolor[rgb]{0.38,0.63,0.69}{\textbf{\textit{{#1}}}}}
    \newcommand{\CommentVarTok}[1]{\textcolor[rgb]{0.38,0.63,0.69}{\textbf{\textit{{#1}}}}}
    \newcommand{\VariableTok}[1]{\textcolor[rgb]{0.10,0.09,0.49}{{#1}}}
    \newcommand{\ControlFlowTok}[1]{\textcolor[rgb]{0.00,0.44,0.13}{\textbf{{#1}}}}
    \newcommand{\OperatorTok}[1]{\textcolor[rgb]{0.40,0.40,0.40}{{#1}}}
    \newcommand{\BuiltInTok}[1]{{#1}}
    \newcommand{\ExtensionTok}[1]{{#1}}
    \newcommand{\PreprocessorTok}[1]{\textcolor[rgb]{0.74,0.48,0.00}{{#1}}}
    \newcommand{\AttributeTok}[1]{\textcolor[rgb]{0.49,0.56,0.16}{{#1}}}
    \newcommand{\InformationTok}[1]{\textcolor[rgb]{0.38,0.63,0.69}{\textbf{\textit{{#1}}}}}
    \newcommand{\WarningTok}[1]{\textcolor[rgb]{0.38,0.63,0.69}{\textbf{\textit{{#1}}}}}
    
    
    % Define a nice break command that doesn't care if a line doesn't already
    % exist.
    \def\br{\hspace*{\fill} \\* }
    % Math Jax compatability definitions
    \def\gt{>}
    \def\lt{<}
    % Document parameters
    \title{Lesson 5- Crawl and scrape}
    
    
    

    % Pygments definitions
    
\makeatletter
\def\PY@reset{\let\PY@it=\relax \let\PY@bf=\relax%
    \let\PY@ul=\relax \let\PY@tc=\relax%
    \let\PY@bc=\relax \let\PY@ff=\relax}
\def\PY@tok#1{\csname PY@tok@#1\endcsname}
\def\PY@toks#1+{\ifx\relax#1\empty\else%
    \PY@tok{#1}\expandafter\PY@toks\fi}
\def\PY@do#1{\PY@bc{\PY@tc{\PY@ul{%
    \PY@it{\PY@bf{\PY@ff{#1}}}}}}}
\def\PY#1#2{\PY@reset\PY@toks#1+\relax+\PY@do{#2}}

\expandafter\def\csname PY@tok@w\endcsname{\def\PY@tc##1{\textcolor[rgb]{0.73,0.73,0.73}{##1}}}
\expandafter\def\csname PY@tok@c\endcsname{\let\PY@it=\textit\def\PY@tc##1{\textcolor[rgb]{0.25,0.50,0.50}{##1}}}
\expandafter\def\csname PY@tok@cp\endcsname{\def\PY@tc##1{\textcolor[rgb]{0.74,0.48,0.00}{##1}}}
\expandafter\def\csname PY@tok@k\endcsname{\let\PY@bf=\textbf\def\PY@tc##1{\textcolor[rgb]{0.00,0.50,0.00}{##1}}}
\expandafter\def\csname PY@tok@kp\endcsname{\def\PY@tc##1{\textcolor[rgb]{0.00,0.50,0.00}{##1}}}
\expandafter\def\csname PY@tok@kt\endcsname{\def\PY@tc##1{\textcolor[rgb]{0.69,0.00,0.25}{##1}}}
\expandafter\def\csname PY@tok@o\endcsname{\def\PY@tc##1{\textcolor[rgb]{0.40,0.40,0.40}{##1}}}
\expandafter\def\csname PY@tok@ow\endcsname{\let\PY@bf=\textbf\def\PY@tc##1{\textcolor[rgb]{0.67,0.13,1.00}{##1}}}
\expandafter\def\csname PY@tok@nb\endcsname{\def\PY@tc##1{\textcolor[rgb]{0.00,0.50,0.00}{##1}}}
\expandafter\def\csname PY@tok@nf\endcsname{\def\PY@tc##1{\textcolor[rgb]{0.00,0.00,1.00}{##1}}}
\expandafter\def\csname PY@tok@nc\endcsname{\let\PY@bf=\textbf\def\PY@tc##1{\textcolor[rgb]{0.00,0.00,1.00}{##1}}}
\expandafter\def\csname PY@tok@nn\endcsname{\let\PY@bf=\textbf\def\PY@tc##1{\textcolor[rgb]{0.00,0.00,1.00}{##1}}}
\expandafter\def\csname PY@tok@ne\endcsname{\let\PY@bf=\textbf\def\PY@tc##1{\textcolor[rgb]{0.82,0.25,0.23}{##1}}}
\expandafter\def\csname PY@tok@nv\endcsname{\def\PY@tc##1{\textcolor[rgb]{0.10,0.09,0.49}{##1}}}
\expandafter\def\csname PY@tok@no\endcsname{\def\PY@tc##1{\textcolor[rgb]{0.53,0.00,0.00}{##1}}}
\expandafter\def\csname PY@tok@nl\endcsname{\def\PY@tc##1{\textcolor[rgb]{0.63,0.63,0.00}{##1}}}
\expandafter\def\csname PY@tok@ni\endcsname{\let\PY@bf=\textbf\def\PY@tc##1{\textcolor[rgb]{0.60,0.60,0.60}{##1}}}
\expandafter\def\csname PY@tok@na\endcsname{\def\PY@tc##1{\textcolor[rgb]{0.49,0.56,0.16}{##1}}}
\expandafter\def\csname PY@tok@nt\endcsname{\let\PY@bf=\textbf\def\PY@tc##1{\textcolor[rgb]{0.00,0.50,0.00}{##1}}}
\expandafter\def\csname PY@tok@nd\endcsname{\def\PY@tc##1{\textcolor[rgb]{0.67,0.13,1.00}{##1}}}
\expandafter\def\csname PY@tok@s\endcsname{\def\PY@tc##1{\textcolor[rgb]{0.73,0.13,0.13}{##1}}}
\expandafter\def\csname PY@tok@sd\endcsname{\let\PY@it=\textit\def\PY@tc##1{\textcolor[rgb]{0.73,0.13,0.13}{##1}}}
\expandafter\def\csname PY@tok@si\endcsname{\let\PY@bf=\textbf\def\PY@tc##1{\textcolor[rgb]{0.73,0.40,0.53}{##1}}}
\expandafter\def\csname PY@tok@se\endcsname{\let\PY@bf=\textbf\def\PY@tc##1{\textcolor[rgb]{0.73,0.40,0.13}{##1}}}
\expandafter\def\csname PY@tok@sr\endcsname{\def\PY@tc##1{\textcolor[rgb]{0.73,0.40,0.53}{##1}}}
\expandafter\def\csname PY@tok@ss\endcsname{\def\PY@tc##1{\textcolor[rgb]{0.10,0.09,0.49}{##1}}}
\expandafter\def\csname PY@tok@sx\endcsname{\def\PY@tc##1{\textcolor[rgb]{0.00,0.50,0.00}{##1}}}
\expandafter\def\csname PY@tok@m\endcsname{\def\PY@tc##1{\textcolor[rgb]{0.40,0.40,0.40}{##1}}}
\expandafter\def\csname PY@tok@gh\endcsname{\let\PY@bf=\textbf\def\PY@tc##1{\textcolor[rgb]{0.00,0.00,0.50}{##1}}}
\expandafter\def\csname PY@tok@gu\endcsname{\let\PY@bf=\textbf\def\PY@tc##1{\textcolor[rgb]{0.50,0.00,0.50}{##1}}}
\expandafter\def\csname PY@tok@gd\endcsname{\def\PY@tc##1{\textcolor[rgb]{0.63,0.00,0.00}{##1}}}
\expandafter\def\csname PY@tok@gi\endcsname{\def\PY@tc##1{\textcolor[rgb]{0.00,0.63,0.00}{##1}}}
\expandafter\def\csname PY@tok@gr\endcsname{\def\PY@tc##1{\textcolor[rgb]{1.00,0.00,0.00}{##1}}}
\expandafter\def\csname PY@tok@ge\endcsname{\let\PY@it=\textit}
\expandafter\def\csname PY@tok@gs\endcsname{\let\PY@bf=\textbf}
\expandafter\def\csname PY@tok@gp\endcsname{\let\PY@bf=\textbf\def\PY@tc##1{\textcolor[rgb]{0.00,0.00,0.50}{##1}}}
\expandafter\def\csname PY@tok@go\endcsname{\def\PY@tc##1{\textcolor[rgb]{0.53,0.53,0.53}{##1}}}
\expandafter\def\csname PY@tok@gt\endcsname{\def\PY@tc##1{\textcolor[rgb]{0.00,0.27,0.87}{##1}}}
\expandafter\def\csname PY@tok@err\endcsname{\def\PY@bc##1{\setlength{\fboxsep}{0pt}\fcolorbox[rgb]{1.00,0.00,0.00}{1,1,1}{\strut ##1}}}
\expandafter\def\csname PY@tok@kc\endcsname{\let\PY@bf=\textbf\def\PY@tc##1{\textcolor[rgb]{0.00,0.50,0.00}{##1}}}
\expandafter\def\csname PY@tok@kd\endcsname{\let\PY@bf=\textbf\def\PY@tc##1{\textcolor[rgb]{0.00,0.50,0.00}{##1}}}
\expandafter\def\csname PY@tok@kn\endcsname{\let\PY@bf=\textbf\def\PY@tc##1{\textcolor[rgb]{0.00,0.50,0.00}{##1}}}
\expandafter\def\csname PY@tok@kr\endcsname{\let\PY@bf=\textbf\def\PY@tc##1{\textcolor[rgb]{0.00,0.50,0.00}{##1}}}
\expandafter\def\csname PY@tok@bp\endcsname{\def\PY@tc##1{\textcolor[rgb]{0.00,0.50,0.00}{##1}}}
\expandafter\def\csname PY@tok@fm\endcsname{\def\PY@tc##1{\textcolor[rgb]{0.00,0.00,1.00}{##1}}}
\expandafter\def\csname PY@tok@vc\endcsname{\def\PY@tc##1{\textcolor[rgb]{0.10,0.09,0.49}{##1}}}
\expandafter\def\csname PY@tok@vg\endcsname{\def\PY@tc##1{\textcolor[rgb]{0.10,0.09,0.49}{##1}}}
\expandafter\def\csname PY@tok@vi\endcsname{\def\PY@tc##1{\textcolor[rgb]{0.10,0.09,0.49}{##1}}}
\expandafter\def\csname PY@tok@vm\endcsname{\def\PY@tc##1{\textcolor[rgb]{0.10,0.09,0.49}{##1}}}
\expandafter\def\csname PY@tok@sa\endcsname{\def\PY@tc##1{\textcolor[rgb]{0.73,0.13,0.13}{##1}}}
\expandafter\def\csname PY@tok@sb\endcsname{\def\PY@tc##1{\textcolor[rgb]{0.73,0.13,0.13}{##1}}}
\expandafter\def\csname PY@tok@sc\endcsname{\def\PY@tc##1{\textcolor[rgb]{0.73,0.13,0.13}{##1}}}
\expandafter\def\csname PY@tok@dl\endcsname{\def\PY@tc##1{\textcolor[rgb]{0.73,0.13,0.13}{##1}}}
\expandafter\def\csname PY@tok@s2\endcsname{\def\PY@tc##1{\textcolor[rgb]{0.73,0.13,0.13}{##1}}}
\expandafter\def\csname PY@tok@sh\endcsname{\def\PY@tc##1{\textcolor[rgb]{0.73,0.13,0.13}{##1}}}
\expandafter\def\csname PY@tok@s1\endcsname{\def\PY@tc##1{\textcolor[rgb]{0.73,0.13,0.13}{##1}}}
\expandafter\def\csname PY@tok@mb\endcsname{\def\PY@tc##1{\textcolor[rgb]{0.40,0.40,0.40}{##1}}}
\expandafter\def\csname PY@tok@mf\endcsname{\def\PY@tc##1{\textcolor[rgb]{0.40,0.40,0.40}{##1}}}
\expandafter\def\csname PY@tok@mh\endcsname{\def\PY@tc##1{\textcolor[rgb]{0.40,0.40,0.40}{##1}}}
\expandafter\def\csname PY@tok@mi\endcsname{\def\PY@tc##1{\textcolor[rgb]{0.40,0.40,0.40}{##1}}}
\expandafter\def\csname PY@tok@il\endcsname{\def\PY@tc##1{\textcolor[rgb]{0.40,0.40,0.40}{##1}}}
\expandafter\def\csname PY@tok@mo\endcsname{\def\PY@tc##1{\textcolor[rgb]{0.40,0.40,0.40}{##1}}}
\expandafter\def\csname PY@tok@ch\endcsname{\let\PY@it=\textit\def\PY@tc##1{\textcolor[rgb]{0.25,0.50,0.50}{##1}}}
\expandafter\def\csname PY@tok@cm\endcsname{\let\PY@it=\textit\def\PY@tc##1{\textcolor[rgb]{0.25,0.50,0.50}{##1}}}
\expandafter\def\csname PY@tok@cpf\endcsname{\let\PY@it=\textit\def\PY@tc##1{\textcolor[rgb]{0.25,0.50,0.50}{##1}}}
\expandafter\def\csname PY@tok@c1\endcsname{\let\PY@it=\textit\def\PY@tc##1{\textcolor[rgb]{0.25,0.50,0.50}{##1}}}
\expandafter\def\csname PY@tok@cs\endcsname{\let\PY@it=\textit\def\PY@tc##1{\textcolor[rgb]{0.25,0.50,0.50}{##1}}}

\def\PYZbs{\char`\\}
\def\PYZus{\char`\_}
\def\PYZob{\char`\{}
\def\PYZcb{\char`\}}
\def\PYZca{\char`\^}
\def\PYZam{\char`\&}
\def\PYZlt{\char`\<}
\def\PYZgt{\char`\>}
\def\PYZsh{\char`\#}
\def\PYZpc{\char`\%}
\def\PYZdl{\char`\$}
\def\PYZhy{\char`\-}
\def\PYZsq{\char`\'}
\def\PYZdq{\char`\"}
\def\PYZti{\char`\~}
% for compatibility with earlier versions
\def\PYZat{@}
\def\PYZlb{[}
\def\PYZrb{]}
\makeatother


    % Exact colors from NB
    \definecolor{incolor}{rgb}{0.0, 0.0, 0.5}
    \definecolor{outcolor}{rgb}{0.545, 0.0, 0.0}



    
    % Prevent overflowing lines due to hard-to-break entities
    \sloppy 
    % Setup hyperref package
    \hypersetup{
      breaklinks=true,  % so long urls are correctly broken across lines
      colorlinks=true,
      urlcolor=urlcolor,
      linkcolor=linkcolor,
      citecolor=citecolor,
      }
    % Slightly bigger margins than the latex defaults
    
    \geometry{verbose,tmargin=1in,bmargin=1in,lmargin=1in,rmargin=1in}
    
    

    \begin{document}
    
    
    \maketitle
    
    

    
    \section{Lesson 5 - Crawl and Scrape}\label{lesson-5---crawl-and-scrape}

\subsection{1. Making the request}\label{making-the-request}

    \subsubsection{1.2 Using 'requests' module}\label{using-requests-module}

Use the requests module to make a HTTP request to
http://www.tripadvisor.com - Check the status of the request - Display
the response header information

    \begin{Verbatim}[commandchars=\\\{\}]
{\color{incolor}In [{\color{incolor} }]:} \PY{k+kn}{import} \PY{n+nn}{requests} 
        \PY{n}{url} \PY{o}{=} \PY{l+s+s1}{\PYZsq{}}\PY{l+s+s1}{http://www.tripadvisor.com/}\PY{l+s+s1}{\PYZsq{}}
        \PY{n}{response} \PY{o}{=} \PY{n}{requests}\PY{o}{.}\PY{n}{get}\PY{p}{(}\PY{n}{url}\PY{p}{)}
        
        \PY{n+nb}{print}\PY{p}{(}\PY{n}{response}\PY{o}{.}\PY{n}{status\PYZus{}code}\PY{p}{)}
        \PY{c+c1}{\PYZsh{}print(response.headers)}
\end{Verbatim}


    \subsubsection{1.3 Get the HTML content from the
website}\label{get-the-html-content-from-the-website}

    \begin{Verbatim}[commandchars=\\\{\}]
{\color{incolor}In [{\color{incolor} }]:} \PY{k+kn}{import} \PY{n+nn}{requests} 
        \PY{n}{url} \PY{o}{=} \PY{l+s+s1}{\PYZsq{}}\PY{l+s+s1}{http://tripadvisor.com}\PY{l+s+s1}{\PYZsq{}}
        \PY{n}{response} \PY{o}{=} \PY{n}{requests}\PY{o}{.}\PY{n}{get}\PY{p}{(}\PY{n}{url}\PY{p}{)}
        
        \PY{k}{if} \PY{n}{response}\PY{o}{.}\PY{n}{status\PYZus{}code} \PY{o}{==} \PY{l+m+mi}{200}\PY{p}{:}
            \PY{n+nb}{print}\PY{p}{(}\PY{n}{response}\PY{o}{.}\PY{n}{status\PYZus{}code}\PY{p}{)}
        \PY{k}{else}\PY{p}{:}
            \PY{n+nb}{print}\PY{p}{(}\PY{l+s+s1}{\PYZsq{}}\PY{l+s+s1}{Failed to get a response from the url. Error code: }\PY{l+s+s1}{\PYZsq{}}\PY{p}{,}\PY{n}{resp}\PY{o}{.}\PY{n}{status\PYZus{}code} \PY{p}{)}
\end{Verbatim}


    \subsubsection{1.4 Get the '/robots.txt' file
contents}\label{get-the-robots.txt-file-contents}

    \begin{Verbatim}[commandchars=\\\{\}]
{\color{incolor}In [{\color{incolor} }]:} \PY{k+kn}{import} \PY{n+nn}{requests} 
        \PY{n}{url} \PY{o}{=} \PY{l+s+s1}{\PYZsq{}}\PY{l+s+s1}{http://www.tripadvisor.com/robots.txt}\PY{l+s+s1}{\PYZsq{}}
        \PY{n}{response} \PY{o}{=} \PY{n}{requests}\PY{o}{.}\PY{n}{get}\PY{p}{(}\PY{n}{url}\PY{p}{)}
        
        \PY{k}{if} \PY{n}{response}\PY{o}{.}\PY{n}{status\PYZus{}code} \PY{o}{==} \PY{l+m+mi}{200}\PY{p}{:}
            \PY{n+nb}{print}\PY{p}{(}\PY{n}{response}\PY{o}{.}\PY{n}{status\PYZus{}code}\PY{p}{)}
            \PY{c+c1}{\PYZsh{}print(response.text)}
        \PY{k}{else}\PY{p}{:}
            \PY{n+nb}{print}\PY{p}{(}\PY{l+s+s1}{\PYZsq{}}\PY{l+s+s1}{Failed to get a response from the url. Error code: }\PY{l+s+s1}{\PYZsq{}}\PY{p}{,}\PY{n}{resp}\PY{o}{.}\PY{n}{status\PYZus{}code} \PY{p}{)}
\end{Verbatim}


    \subsection{2. Scraping websites}\label{scraping-websites}

Sometimes, you may want a little bit of information - a movie rating,
stock price, or product availability - but the information is available
only in HTML pages, surrounded by ads and extraneous content.

To do this we build an automated web fetcher called a crawler or spider.
After the HTML contents have been retrived from the remote web servers,
a scraper parses it to find the needle in the haystack.

    \subsubsection{2.1 BeautifulSoup Module}\label{beautifulsoup-module}

    The BS4 module can be used for searching a webpage (HTML file) and
pulling required data from it. It does three things to make a HTML page
searchable- * First, converts the HTML page to Unicode, and HTML
entities are converted to Unicode characters * Second, parses (analyses)
the HTML page using the best available parser. It will use an HTML
parser unless you specifically tell it to use an XML parser * Finally
transforms a complex HTML document into a complex tree of Python
objects.

This module takes the HTML page and creates four kinds of objects: Tag,
NavigableString, BeautifulSoup, and Comment. * The
\textbf{\emph{BeautifulSoup}} object itself represents the webpage as a
whole * A \textbf{\emph{Tag}} object corresponds to an XML or HTML tag
in the webpage * The \textbf{\emph{NavigableString}} class to contains
the bit of text within a tag

https://www.crummy.com/software/BeautifulSoup/bs4/doc/

    \begin{Verbatim}[commandchars=\\\{\}]
{\color{incolor}In [{\color{incolor} }]:} \PY{o}{\PYZlt{}}\PY{n}{h1} \PY{n+nb}{id}\PY{o}{=}\PY{l+s+s2}{\PYZdq{}}\PY{l+s+s2}{HEADING}\PY{l+s+s2}{\PYZdq{}} \PY{n+nb}{property}\PY{o}{=}\PY{l+s+s2}{\PYZdq{}}\PY{l+s+s2}{name}\PY{l+s+s2}{\PYZdq{}} \PY{n}{class}\PY{o}{=}\PY{l+s+s2}{\PYZdq{}}\PY{l+s+s2}{heading\PYZus{}name   }\PY{l+s+s2}{\PYZdq{}}\PY{o}{\PYZgt{}}
            \PY{o}{\PYZlt{}}\PY{n}{div} \PY{n}{class}\PY{o}{=}\PY{l+s+s2}{\PYZdq{}}\PY{l+s+s2}{heading\PYZus{}height}\PY{l+s+s2}{\PYZdq{}}\PY{o}{\PYZgt{}}\PY{o}{\PYZlt{}}\PY{o}{/}\PY{n}{div}\PY{o}{\PYZgt{}}
             \PY{l+s+s2}{\PYZdq{}}
             \PY{n}{Le} \PY{n}{Jardin} \PY{n}{Napolitain}
             \PY{l+s+s2}{\PYZdq{}}
        \PY{o}{\PYZlt{}}\PY{o}{/}\PY{n}{h1}\PY{o}{\PYZgt{}}
\end{Verbatim}


    \paragraph{Step 1: Making the soup}\label{step-1-making-the-soup}

First we need to use the BeautifulSoup module to parse the HTML data
into Python readable Unicode Text format.

    \emph{Let us write the code to parse a html page. We will use the trip
advisor URL for an infamous restaurant -
https://www.tripadvisor.com/Restaurant\_Review-g187147-d1751525-Reviews-Cafe\_Le\_Dome-Paris\_Ile\_de\_France.html
}

    \begin{Verbatim}[commandchars=\\\{\}]
{\color{incolor}In [{\color{incolor} }]:} \PY{k+kn}{import} \PY{n+nn}{requests}
        \PY{k+kn}{from} \PY{n+nn}{bs4} \PY{k}{import} \PY{n}{BeautifulSoup}
        \PY{n}{scrape\PYZus{}url} \PY{o}{=} \PY{l+s+s1}{\PYZsq{}}\PY{l+s+s1}{https://www.tripadvisor.com/Restaurant\PYZus{}Review\PYZhy{}g187147\PYZhy{}d1751525\PYZhy{}Reviews\PYZhy{}Cafe\PYZus{}Le\PYZus{}Dome\PYZhy{}Paris\PYZus{}Ile\PYZus{}de\PYZus{}France.html}\PY{l+s+s1}{\PYZsq{}}  
        \PY{n}{response} \PY{o}{=} \PY{n}{requests}\PY{o}{.}\PY{n}{get}\PY{p}{(}\PY{n}{scrape\PYZus{}url}\PY{p}{)} 
        \PY{n+nb}{print}\PY{p}{(}\PY{n}{response}\PY{o}{.}\PY{n}{status\PYZus{}code}\PY{p}{)}
        
        \PY{k}{if} \PY{n}{response}\PY{o}{.}\PY{n}{status\PYZus{}code} \PY{o}{==} \PY{l+m+mi}{200}\PY{p}{:}
            \PY{n}{soup} \PY{o}{=} \PY{n}{BeautifulSoup}\PY{p}{(}\PY{n}{response}\PY{o}{.}\PY{n}{text}\PY{p}{,} \PY{l+s+s1}{\PYZsq{}}\PY{l+s+s1}{html.parser}\PY{l+s+s1}{\PYZsq{}}\PY{p}{)} \PY{c+c1}{\PYZsh{} Soup}
            \PY{c+c1}{\PYZsh{}print(soup)}
\end{Verbatim}


    \paragraph{Step 2: Inspect the element you want to
scrape}\label{step-2-inspect-the-element-you-want-to-scrape}

In this step we will inspect the HTML data of the website to understand
the tags and attributes that matches the element. Let us inspect the
HTML data of the URL and understand where (under which tag) the review
data is located.

    \begin{Verbatim}[commandchars=\\\{\}]
{\color{incolor}In [{\color{incolor} }]:} \PY{o}{\PYZlt{}}\PY{n}{div} \PY{n}{class}\PY{o}{=}\PY{l+s+s2}{\PYZdq{}}\PY{l+s+s2}{entry}\PY{l+s+s2}{\PYZdq{}}\PY{o}{\PYZgt{}}
            \PY{o}{\PYZlt{}}\PY{n}{p} \PY{n}{class}\PY{o}{=}\PY{l+s+s2}{\PYZdq{}}\PY{l+s+s2}{partial\PYZus{}entry}\PY{l+s+s2}{\PYZdq{}}\PY{o}{\PYZgt{}}
            \PY{n}{Popped} \PY{o+ow}{in} \PY{n}{on} \PY{n}{way} \PY{n}{to} \PY{n}{Eiffel} \PY{n}{Tower} \PY{k}{for} \PY{n}{lunch}\PY{p}{,} \PY{n}{big} \PY{n}{mistake}\PY{o}{.} 
            \PY{n}{Pizza} \PY{n}{was} \PY{n}{disgusting} \PY{o+ow}{and} \PY{n}{service} \PY{n}{was} \PY{n}{poor}\PY{o}{.} 
            \PY{n}{It}\PY{err}{’}\PY{n}{s} \PY{n}{a} \PY{n}{shame} \PY{n}{Trip} \PY{n}{Advisor} \PY{n}{don}\PY{err}{’}\PY{n}{t} \PY{n}{let} \PY{n}{you} \PY{n}{score} \PY{n}{venues} \PY{n}{zero}\PY{o}{.}\PY{o}{.}\PY{o}{.}\PY{o}{.}
            \PY{o}{\PYZlt{}}\PY{n}{span} \PY{n}{class}\PY{o}{=}\PY{l+s+s2}{\PYZdq{}}\PY{l+s+s2}{taLnk ulBlueLinks}\PY{l+s+s2}{\PYZdq{}} \PY{n}{onclick}\PY{o}{=}\PY{l+s+s2}{\PYZdq{}}\PY{l+s+s2}{widgetEvCall(}\PY{l+s+s2}{\PYZsq{}}\PY{l+s+s2}{handlers.clickExpand}\PY{l+s+s2}{\PYZsq{}}\PY{l+s+s2}{,event,this);}\PY{l+s+s2}{\PYZdq{}}\PY{o}{\PYZgt{}}\PY{n}{More}
            \PY{o}{\PYZlt{}}\PY{o}{/}\PY{n}{span}\PY{o}{\PYZgt{}}
            \PY{o}{\PYZlt{}}\PY{o}{/}\PY{n}{p}\PY{o}{\PYZgt{}}
        \PY{o}{\PYZlt{}}\PY{o}{/}\PY{n}{div}\PY{o}{\PYZgt{}}
\end{Verbatim}


    \paragraph{Step 3: Searching the soup for the
data}\label{step-3-searching-the-soup-for-the-data}

Beautiful Soup defines a lot of methods for searching the parse tree
(soup), the two most popular methods are: find() and find\_all().

The simplest filter is a tag. Pass a tag to a search method and
Beautiful Soup will perform a match against that exact string.

    \emph{Let us try and find all the \textless{} p \textgreater{}
(paragraph) tags in the soup:}

    \begin{Verbatim}[commandchars=\\\{\}]
{\color{incolor}In [{\color{incolor} }]:} \PY{k+kn}{import} \PY{n+nn}{requests}
        \PY{k+kn}{from} \PY{n+nn}{bs4} \PY{k}{import} \PY{n}{BeautifulSoup}
        
        \PY{k}{def} \PY{n+nf}{scrapecontent}\PY{p}{(}\PY{n}{url}\PY{p}{)}\PY{p}{:}
            \PY{l+s+sd}{\PYZdq{}\PYZdq{}\PYZdq{}This function parses the HTML page representing the url using the BeautifulSoup module}
        \PY{l+s+sd}{    and returns the created python readable data structure (soup)\PYZdq{}\PYZdq{}\PYZdq{}}
            \PY{n}{scrape\PYZus{}response} \PY{o}{=} \PY{n}{requests}\PY{o}{.}\PY{n}{get}\PY{p}{(}\PY{n}{url}\PY{p}{)} 
            \PY{n+nb}{print}\PY{p}{(}\PY{n}{scrape\PYZus{}response}\PY{o}{.}\PY{n}{status\PYZus{}code}\PY{p}{)}
        
            \PY{k}{if} \PY{n}{scrape\PYZus{}response}\PY{o}{.}\PY{n}{status\PYZus{}code} \PY{o}{==} \PY{l+m+mi}{200}\PY{p}{:}
                \PY{n}{soup} \PY{o}{=} \PY{n}{BeautifulSoup}\PY{p}{(}\PY{n}{scrape\PYZus{}response}\PY{o}{.}\PY{n}{text}\PY{p}{,} \PY{l+s+s1}{\PYZsq{}}\PY{l+s+s1}{html.parser}\PY{l+s+s1}{\PYZsq{}}\PY{p}{)}
                \PY{k}{return} \PY{n}{soup}
            \PY{k}{else}\PY{p}{:}
                \PY{n+nb}{print}\PY{p}{(}\PY{l+s+s1}{\PYZsq{}}\PY{l+s+s1}{Error accessing url : }\PY{l+s+s1}{\PYZsq{}}\PY{p}{,}\PY{n}{scrape\PYZus{}response}\PY{o}{.}\PY{n}{status\PYZus{}code}\PY{p}{)}
                \PY{k}{return} \PY{k+kc}{None}
        
        \PY{k}{def} \PY{n+nf}{main}\PY{p}{(}\PY{p}{)}\PY{p}{:}
            \PY{n}{scrape\PYZus{}url} \PY{o}{=} \PY{l+s+s1}{\PYZsq{}}\PY{l+s+s1}{https://www.tripadvisor.com/Restaurant\PYZus{}Review\PYZhy{}g187147\PYZhy{}d1751525\PYZhy{}Reviews\PYZhy{}Cafe\PYZus{}Le\PYZus{}Dome\PYZhy{}Paris\PYZus{}Ile\PYZus{}de\PYZus{}France.html}\PY{l+s+s1}{\PYZsq{}}  
            \PY{n}{ret\PYZus{}soup} \PY{o}{=} \PY{n}{scrapecontent}\PY{p}{(}\PY{n}{scrape\PYZus{}url}\PY{p}{)}
            \PY{k}{if} \PY{n}{ret\PYZus{}soup}\PY{p}{:}
                \PY{k}{for} \PY{n}{review} \PY{o+ow}{in} \PY{n}{ret\PYZus{}soup}\PY{o}{.}\PY{n}{find\PYZus{}all}\PY{p}{(}\PY{l+s+s1}{\PYZsq{}}\PY{l+s+s1}{p}\PY{l+s+s1}{\PYZsq{}}\PY{p}{,} \PY{n}{class\PYZus{}}\PY{o}{=}\PY{l+s+s1}{\PYZsq{}}\PY{l+s+s1}{partial\PYZus{}entry}\PY{l+s+s1}{\PYZsq{}}\PY{p}{)}\PY{p}{:}        
                    \PY{n+nb}{print}\PY{p}{(}\PY{n}{review}\PY{o}{.}\PY{n}{text}\PY{p}{)} \PY{c+c1}{\PYZsh{}We are interested only in the text data, since the reviews are stored as text}
        \PY{n}{main}\PY{p}{(}\PY{p}{)}
\end{Verbatim}


    \paragraph{Step 4: Enable pagination}\label{step-4-enable-pagination}

Automatically access subsequent pages

    \begin{Verbatim}[commandchars=\\\{\}]
{\color{incolor}In [{\color{incolor} }]:} \PY{k+kn}{import} \PY{n+nn}{requests}
        \PY{k+kn}{from} \PY{n+nn}{bs4} \PY{k}{import} \PY{n}{BeautifulSoup}
        
        \PY{k}{def} \PY{n+nf}{scrapecontent}\PY{p}{(}\PY{n}{url}\PY{p}{)}\PY{p}{:}
            \PY{l+s+sd}{\PYZdq{}\PYZdq{}\PYZdq{}This function parses the HTML page representing the url using the BeautifulSoup module}
        \PY{l+s+sd}{    and returns the created python readable data structure (soup)\PYZdq{}\PYZdq{}\PYZdq{}}
            \PY{n}{scrape\PYZus{}response} \PY{o}{=} \PY{n}{requests}\PY{o}{.}\PY{n}{get}\PY{p}{(}\PY{n}{url}\PY{p}{)} 
            \PY{n+nb}{print}\PY{p}{(}\PY{n}{scrape\PYZus{}response}\PY{o}{.}\PY{n}{status\PYZus{}code}\PY{p}{)}
        
            \PY{k}{if} \PY{n}{scrape\PYZus{}response}\PY{o}{.}\PY{n}{status\PYZus{}code} \PY{o}{==} \PY{l+m+mi}{200}\PY{p}{:}
                \PY{n}{soup} \PY{o}{=} \PY{n}{BeautifulSoup}\PY{p}{(}\PY{n}{scrape\PYZus{}response}\PY{o}{.}\PY{n}{text}\PY{p}{,} \PY{l+s+s1}{\PYZsq{}}\PY{l+s+s1}{html.parser}\PY{l+s+s1}{\PYZsq{}}\PY{p}{)}
                \PY{k}{return} \PY{n}{soup}
            \PY{k}{else}\PY{p}{:}
                \PY{n+nb}{print}\PY{p}{(}\PY{l+s+s1}{\PYZsq{}}\PY{l+s+s1}{Error accessing url : }\PY{l+s+s1}{\PYZsq{}}\PY{p}{,}\PY{n}{scrape\PYZus{}response}\PY{o}{.}\PY{n}{status\PYZus{}code}\PY{p}{)}
                \PY{k}{return} \PY{k+kc}{None}
        
        \PY{k}{def} \PY{n+nf}{main}\PY{p}{(}\PY{p}{)}\PY{p}{:}
            \PY{n}{page\PYZus{}no} \PY{o}{=} \PY{l+m+mi}{0}
        
            \PY{k}{while}\PY{p}{(}\PY{n}{page\PYZus{}no} \PY{o}{\PYZlt{}} \PY{l+m+mi}{60}\PY{p}{)}\PY{p}{:}
                \PY{n}{scrape\PYZus{}url} \PY{o}{=} \PY{l+s+s1}{\PYZsq{}}\PY{l+s+s1}{https://www.tripadvisor.com/Restaurant\PYZus{}Review\PYZhy{}g187147\PYZhy{}d1751525\PYZhy{}Reviews\PYZhy{}or}\PY{l+s+s1}{\PYZsq{}}\PY{o}{+}\PY{n+nb}{str}\PY{p}{(}\PY{n}{page\PYZus{}no}\PY{p}{)}\PY{o}{+}\PY{l+s+s1}{\PYZsq{}}\PY{l+s+s1}{\PYZhy{}Cafe\PYZus{}Le\PYZus{}Dome\PYZhy{}Paris\PYZus{}Ile\PYZus{}de\PYZus{}France.html}\PY{l+s+s1}{\PYZsq{}}  
                \PY{n}{ret\PYZus{}soup} \PY{o}{=} \PY{n}{scrapecontent}\PY{p}{(}\PY{n}{scrape\PYZus{}url}\PY{p}{)}
                \PY{k}{if} \PY{n}{ret\PYZus{}soup}\PY{p}{:}
                    \PY{k}{for} \PY{n}{review} \PY{o+ow}{in} \PY{n}{ret\PYZus{}soup}\PY{o}{.}\PY{n}{find\PYZus{}all}\PY{p}{(}\PY{l+s+s1}{\PYZsq{}}\PY{l+s+s1}{p}\PY{l+s+s1}{\PYZsq{}}\PY{p}{,} \PY{n}{class\PYZus{}}\PY{o}{=}\PY{l+s+s1}{\PYZsq{}}\PY{l+s+s1}{partial\PYZus{}entry}\PY{l+s+s1}{\PYZsq{}}\PY{p}{)}\PY{p}{:}        
                        \PY{n+nb}{print}\PY{p}{(}\PY{n}{review}\PY{o}{.}\PY{n}{text}\PY{p}{)} \PY{c+c1}{\PYZsh{}We are interested only in the text data, since the reviews are stored as text}
                \PY{n}{page\PYZus{}no} \PY{o}{=} \PY{n}{page\PYZus{}no} \PY{o}{+} \PY{l+m+mi}{10}
                    
        \PY{n}{main}\PY{p}{(}\PY{p}{)}
\end{Verbatim}


    \emph{Using yesterdays sentiment analysis code and the corpus of
sentiment found in the word\_sentiment.csv file, calculate the sentiment
of the reviews.}

    \begin{Verbatim}[commandchars=\\\{\}]
{\color{incolor}In [{\color{incolor} }]:} \PY{c+c1}{\PYZsh{}Enter your code here}
\end{Verbatim}


    \paragraph{Expanding this further}\label{expanding-this-further}

To add additional details we can inspect the tags further and add the
reviewer rating and reviwer details.

    \begin{Verbatim}[commandchars=\\\{\}]
{\color{incolor}In [{\color{incolor} }]:} \PY{k+kn}{import} \PY{n+nn}{requests}
        \PY{k+kn}{from} \PY{n+nn}{bs4} \PY{k}{import} \PY{n}{BeautifulSoup}
        \PY{k}{def} \PY{n+nf}{scrapecontent}\PY{p}{(}\PY{n}{url}\PY{p}{)}\PY{p}{:}
            \PY{l+s+sd}{\PYZdq{}\PYZdq{}\PYZdq{}This function parses the HTML page representing the url using the BeautifulSoup module}
        \PY{l+s+sd}{    and returns the created python readable data structure (soup)\PYZdq{}\PYZdq{}\PYZdq{}}
            \PY{n}{scrape\PYZus{}response} \PY{o}{=} \PY{n}{requests}\PY{o}{.}\PY{n}{get}\PY{p}{(}\PY{n}{url}\PY{p}{)} 
            \PY{n+nb}{print}\PY{p}{(}\PY{n}{scrape\PYZus{}response}\PY{o}{.}\PY{n}{status\PYZus{}code}\PY{p}{)}
        
            \PY{k}{if} \PY{n}{scrape\PYZus{}response}\PY{o}{.}\PY{n}{status\PYZus{}code} \PY{o}{==} \PY{l+m+mi}{200}\PY{p}{:}
                \PY{n}{soup} \PY{o}{=} \PY{n}{BeautifulSoup}\PY{p}{(}\PY{n}{scrape\PYZus{}response}\PY{o}{.}\PY{n}{text}\PY{p}{,} \PY{l+s+s1}{\PYZsq{}}\PY{l+s+s1}{html.parser}\PY{l+s+s1}{\PYZsq{}}\PY{p}{)}
                \PY{k}{return} \PY{n}{soup}
            \PY{k}{else}\PY{p}{:}
                \PY{n+nb}{print}\PY{p}{(}\PY{l+s+s1}{\PYZsq{}}\PY{l+s+s1}{Error accessing url : }\PY{l+s+s1}{\PYZsq{}}\PY{p}{,}\PY{n}{scrape\PYZus{}response}\PY{o}{.}\PY{n}{status\PYZus{}code}\PY{p}{)}
                \PY{k}{return} \PY{k+kc}{None}
        
        \PY{k}{def} \PY{n+nf}{main}\PY{p}{(}\PY{p}{)}\PY{p}{:}
            \PY{n}{scrape\PYZus{}url} \PY{o}{=} \PY{l+s+s1}{\PYZsq{}}\PY{l+s+s1}{https://www.tripadvisor.com/Restaurant\PYZus{}Review\PYZhy{}g187147\PYZhy{}d1751525\PYZhy{}Reviews\PYZhy{}Cafe\PYZus{}Le\PYZus{}Dome\PYZhy{}Paris\PYZus{}Ile\PYZus{}de\PYZus{}France.html}\PY{l+s+s1}{\PYZsq{}}  
            \PY{n}{ret\PYZus{}soup} \PY{o}{=} \PY{n}{scrapecontent}\PY{p}{(}\PY{n}{scrape\PYZus{}url}\PY{p}{)}
            \PY{k}{if} \PY{n}{ret\PYZus{}soup}\PY{p}{:}
                \PY{k}{for} \PY{n}{rev\PYZus{}data} \PY{o+ow}{in} \PY{n}{ret\PYZus{}soup}\PY{o}{.}\PY{n}{find\PYZus{}all}\PY{p}{(}\PY{l+s+s1}{\PYZsq{}}\PY{l+s+s1}{div}\PY{l+s+s1}{\PYZsq{}}\PY{p}{,} \PY{n}{class\PYZus{}}\PY{o}{=} \PY{l+s+s1}{\PYZsq{}}\PY{l+s+s1}{review}\PY{l+s+s1}{\PYZsq{}}\PY{p}{)}\PY{p}{:}
                    \PY{n}{date} \PY{o}{=} \PY{n}{rev\PYZus{}data}\PY{o}{.}\PY{n}{find}\PY{p}{(}\PY{l+s+s1}{\PYZsq{}}\PY{l+s+s1}{span}\PY{l+s+s1}{\PYZsq{}}\PY{p}{,} \PY{n}{class\PYZus{}} \PY{o}{=}\PY{l+s+s1}{\PYZsq{}}\PY{l+s+s1}{ratingDate}\PY{l+s+s1}{\PYZsq{}}\PY{p}{)}\PY{c+c1}{\PYZsh{} Get the date if the review}
                    \PY{n+nb}{print}\PY{p}{(}\PY{n}{date}\PY{o}{.}\PY{n}{text}\PY{p}{)}
                    \PY{n}{review} \PY{o}{=} \PY{n}{rev\PYZus{}data}\PY{o}{.}\PY{n}{find}\PY{p}{(}\PY{l+s+s1}{\PYZsq{}}\PY{l+s+s1}{p}\PY{l+s+s1}{\PYZsq{}}\PY{p}{)} \PY{c+c1}{\PYZsh{} Get the review text}
                    \PY{n+nb}{print}\PY{p}{(}\PY{n}{review}\PY{o}{.}\PY{n}{text}\PY{p}{)}
                    \PY{n}{rating} \PY{o}{=} \PY{n}{rev\PYZus{}data}\PY{o}{.}\PY{n}{find}\PY{p}{(}\PY{l+s+s1}{\PYZsq{}}\PY{l+s+s1}{span}\PY{l+s+s1}{\PYZsq{}}\PY{p}{,}\PY{n}{class\PYZus{}}\PY{o}{=}\PY{l+s+s1}{\PYZsq{}}\PY{l+s+s1}{ui\PYZus{}bubble\PYZus{}rating}\PY{l+s+s1}{\PYZsq{}}\PY{p}{)} \PY{c+c1}{\PYZsh{}Get the rating of the review}
                    \PY{n+nb}{print}\PY{p}{(}\PY{n+nb}{int}\PY{p}{(}\PY{n}{rating}\PY{p}{[}\PY{l+s+s1}{\PYZsq{}}\PY{l+s+s1}{class}\PY{l+s+s1}{\PYZsq{}}\PY{p}{]}\PY{p}{[}\PY{l+m+mi}{1}\PY{p}{]}\PY{p}{[}\PY{l+m+mi}{7}\PY{p}{:}\PY{p}{]}\PY{p}{)}\PY{o}{/}\PY{l+m+mi}{10}\PY{p}{)}
        \PY{n}{main}\PY{p}{(}\PY{p}{)}
\end{Verbatim}


    \emph{Using the review data and the ratings available is there any way
we can improve the corpus of sentiments "word\_sentiment.csv" file?}


    % Add a bibliography block to the postdoc
    
    
    
    \end{document}
